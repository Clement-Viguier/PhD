 \documentclass[english,10pt]{article}
  %\documentclass[english,10pt,twocolumn]{article}
 \usepackage{setspace}
 \onehalfspacing
 \usepackage[utf8]{inputenc}
 \usepackage{lmodern}
 \usepackage[a4paper]{geometry}
 \usepackage{babel}
 \usepackage{wallpaper}
 \usepackage{graphicx} %images
 \usepackage{graphics} %images
 \usepackage{gensymb} %°C
 \usepackage{textcomp}
 \usepackage{multicol}
 \usepackage{lscape}   % paysage
 \usepackage{pdflscape,array,booktabs}%pages du pdf avec tableau en paysage
 \usepackage{geometry}
 \usepackage{multirow} %tableau avec lignes multiples
 \usepackage{fancyhdr}
 %\usepackage[compact]{titlesec} %titres de section compact
 \usepackage{float}
 \usepackage[most]{tcolorbox}
 \definecolor{grey}{gray}{0.95}
 \definecolor{vert}{RGB}{102,194,165}
 \definecolor{red}{RGB}{136,0,21}
 \usepackage{url}
 \usepackage{tocloft}  %table des figures
 \usepackage{hyperref}
% \usepackage{times}
 \usepackage[small,bf,sf,font=sf]{caption}  %gestion des légendes
 \captionsetup[table]{font={stretch=1,sf}}
%Glossaire Need Perl
\usepackage[toc,nonumberlist]{glossaries}
 \makeglossaries
 \usepackage{amsmath,amsfonts,amssymb} %équations, matrices etc...
 \usepackage{cell} %bibliography style
 \usepackage{indentfirst} %indentation des paragraphes en debut de section
% \usepackage[numbered,autolinebreaks,useliterate]{mcode} %insertion de code Matlab framed,
 \geometry{vmargin=2.5cm, hmargin=2cm}
 \hypersetup{colorlinks=true, linkcolor= black, citecolor=black} %gestion lien hypertexte
%\usepackage[rm]{roboto}
 \usepackage[T1]{fontenc}
\usepackage{titlesec}
\usepackage[switch,pagewise]{lineno}

 \newcommand{\auth}{Clément Viguier}
 \newcommand{\model}{\textit{\textbf{MountGrass }}}
 \newcommand{\tlte}{- MountGrass -\\An agent-based model for the exploration of mountain grassland community dynamics}
 \newcommand{\tltesmall}{Plan \date}
 
 \newcommand{\gl}{\hspace{0.6cm}\textbf } %A revoir

\setlength{\columnsep}{20pt}
 %__________________________________________________________
 \newenvironment{changemargin}[2]{\begin{list}{}{%
 \setlength{\topsep}{0pt}%
 \setlength{\leftmargin}{0pt}%
 \setlength{\rightmargin}{0pt}%
 \setlength{\listparindent}{\parindent}%
 \setlength{\itemindent}{\parindent}%
 \setlength{\parsep}{0pt plus 1pt}%
 \addtolength{\leftmargin}{#1}%
 \addtolength{\rightmargin}{#2}%
 }\item }{\end{list}}
 
 \setlength{\parskip}{0em}
 \captionsetup{belowskip=8pt,aboveskip=0pt}
 
 
 %___________________________________________________________
 
%\renewcommand{\familydefault}{\sfdefault}  
 \renewcommand{\rmdefault}{ptm}
 
 \titleformat{\section}{
 	\sffamily\color{red}\Large
 }{\thesection}{1em}{}
 
  \titleformat{\subsection}{
 	\sffamily\color{red}\large
 }{\thesubsection}{1em}{}
 
 % titre du document
 \title{\tlte}
 % auteur du document
 \author{\auth}
 \date{October 2016} %http://en.wikibooks.org/wiki/LaTeX/Title_Creation
 
% \renewcommand{version}{\texttt{MountGrass2.0}}
  
 %Glossaire
 %\newglossaryentry {alios}{name={alios}, description={portion argileuse du sol meuble à solide}}
 %\newacronym{{ademe}{ADEME}{Agence De l'Environnement et de la Maîtrise de l'Énergie}}
 
 
 
 %New paragraph definition avec saut de ligne
\makeatletter
\renewcommand\paragraph{
\@startsection{paragraph}{4}{\z@}
{-1.25ex\@plus -1ex \@minus -.2ex}
{1ex \@plus .2ex}
{\normalfont\normalsize\bfseries}}
\makeatother  
 
 
 %Légende tableaux :
 \newenvironment{captiontable}{
  \begin{small}
 \begin{sffamily}
 \begin{spacing}{1.1}}
 {
 \end{spacing}
 \end{sffamily}
 \end{small}
 }
 
 
 %DEBUT DU DOCUMENT _____________________________________________________________
 \begin{document}
 \pagestyle{fancy}
 \renewcommand\headrulewidth{0.5pt}
 \fancyhead[R]{\color{black}\small\auth - 2016}
 \fancyhead[L]{\color{black}\small\tltesmall}
 \thispagestyle{empty}
% 
% \newpage
% \strut
% \ThisLRCornerWallPaper{1}{../Figures/frontpage.pdf}
% \newpage
 
% \begin{titlepage}
% 
% \maketitle
% \centering
% 
% \thispagestyle{empty}
% \end{titlepage}
%
% \begin{changemargin}{2cm}{2cm}	%Sommaire
% \vspace*{2cm}
% \thispagestyle{empty}
% \tableofcontents
%\thispagestyle{empty} 
% \end{changemargin}
% 
%\newpage
%
% \begin{changemargin}{2cm}{2cm}	%Liste des figures
% \vspace*{2cm}
% \thispagestyle{empty}
%\listoffigures
%\thispagestyle{empty} 
% \end{changemargin}
%\newpage
%
%
% \begin{changemargin}{2cm}{2cm}	%Liste des tableaux
% \vspace*{2cm}
% \thispagestyle{empty}
%\listoftables
%\thispagestyle{empty} 
% \end{changemargin}
%\newpage
%
%\linenumbers
%\modulolinenumbers[5]

\section{Objectives}
\subsection{Generic framework for modelling of plant communities}
\subsection{The effect of phenotypic plasticity on plant community dynamics}
\paragraph{Individual level}
\paragraph{Community response to drought event}

\section{Literature review}
\subsection{Context: mountain grasslands and climate change}

\subsection{Diversity and coexistence}
\paragraph{Effects of diversity}
productivity\\
resistance ?\\
Ecosystem services and complementarity\\

\paragraph{Mechanisms for coexistence}
main theories: niche, neutral, individual based. -> scale and dimension dependant.\\
chesson\\
Spatial and temporal variability\\
trade-off, strategy space, and variability.\\
in the end it's rarely direct interaction but capacity to respond to stress and interect interaction through resource pools.

\subsection{global change and community dynamics: theory and empirical results}
\paragraph{Intraspecific variability}
frame of reference: deep traits vs shallow traits. definition of functional trait.\\
source of intra specific variability: genetic vs ontogeny vs plasticity (epigen) \\
effect on niche and interactions: effect on coexistence\\
-> plasticity a special form of ISV
\paragraph{Understanding phenotypic plasticity}
adaaptive intraspecific variation\\
cost and limits\\
effect on coexistence and community


\subsection{Existing  modelling solutions/approaches to question global change effect on vegetation community}
\paragraph{Modelling vegetation - traits and strategies}
traits \& strategies\\
existing models: a gap to fill\\
coexistence processes



\section{Generic model for plant community dynamics}
Paper 1:
\subsection{Introduction}
\subsection{Strategy space and allocation pools}
Leaf economic spectrum + Shipley

\subsection{Model overview}
\paragraph{pseudo-code and routine}
\paragraph{allocation}
mechanism and stochasticity\\
5 types of allocation\\

\subsection{Plasticity: between species memory and individual experience}
\paragraph{concepts} equilibrium, resource use, resource availability, condition estimation
\paragraph{comparison of different algorithms}
the two sides of the performance/fitness: equilibrium and tissue efficiency\\
age vs biomass.

\subsection{Parameter filtering and sensitivity analysis}
Obj: give confidence in the model, demonstrate is able to reproduce simple growth pattern.\\
Obj2: have a beter idea of plasticity on growth.
growth plastic and non plastic parameter filtering: can we distinguish species thanks to species specific parameters instead of shared parameters.\\
does plasticity make it easier ?\\
Impact of plasticity related parameters.

\subsection{Community dynamics parametrisation}
Obj: demonstrate that the model is able to reproduce community dynamics (as it was designed for).\\
Find parameters that allows coexistence (suggest plasticity should allow a diversity of strategy). SLA and height data. Phytosociology for 10m quadrats.

\section{Model's properties and individuals response}
(Related to the notions cited above, like performance decomposition)

\subsection{Craft a trade-of and phenotypic map}
Can memory be related to strategy and active/structural ratios in shoot and roots ?

\subsection{Niche response}
Obj1: understand how resource use mechanisms and allocation algorithms shape the environmental potential niche in the context of the model.\\
H1: strategy and memory affect niche in two ways if we suppose they are independent: shape and position. Strategy mostly affect shape (width and height) while memory (and so root:shoot ratio) affect mostly position.\\
H1': there is strong link between strategy and memory in the case of optimisation allocation that increase niche height and might reduce its width.\\
Obj2: understand the role of plasticity on the niche and if the effect in the same for all strategies/memories.\\
H2: the plasticity increase niche width but not height (as phenotype is optimum at the center of the niche where memory match the resource availability).

Stability and efficiency trade-off. Niche heigh and width and relationship with the strategy. How does plasticity affect that ? Does it increase the height and widen niches ? What does that mean for coexistence ?\\
Hopefully higher niche would go with unstable niche.

\subsection{Transitivity and competition}
1 vs 1 interactions\\
Is the resource competition transitive ? How does niche widening impact that, does plasticty change competition interaction. Is it related to the trait distance ? (don't think so)

\section{The effect of phenotypic plasticity on plant community dynamics}
Hypothesis on the cumulative effect on niche and interactions.

\subsection{Individual resistance and resilience against drought events}
Amplitude and length of the event :\\
- severity effect reduced by lower tau ?\\
- resistance versus resilience: H0: conservative strategy have higher resistance, H1 : low tau allows for re-equilibrium and increase resistance (low amplitude and long length. H2: high tau allow to avoid dead-end situation during short severe drought (high resilience)
\subsection{Community response to drought event}
coexistence effect vs resistance/resilience effect\\
uniform vs heterogenous (plasticity wise) community response
H1: 


\section{Notes}

\subsection{On modelling}
Frustration: often look obvious, at least it's just logical, there is what we put in...\\

Modelling approach, when not for prediction, what is it about ?
\begin{itemize}
\item building understanding
\item weight mechanisms
\item test hypothesis
\end{itemize}

 \nocite{TitlesOn}
 \bibliographystyle{cell}
 \bibliography{biblio_model_description}


  \end{document}
%  }