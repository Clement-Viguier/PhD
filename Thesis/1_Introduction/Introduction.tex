

%_________________________________________________________________________________
\chapter{Mountain grasslands}
\fwnewthought{Mountain grasslands have an unique beauty drawn by the diversity of flowers colours, the strong contrast between the luxurious green vegetation and the roughness of the naked rocks, and the feeling that living in such places, at the edge of living conditions, is a fight worth fighting. I could illustrate this beauty with thousands pictures and words, and it would certainly convince you that these ecosystems worth spending time studying them to better understand and protect them. I could also describe their role in the economy of alpine regions, currently subject of great modification, and greater to come. But you would not see these rich systems as I see them: as an intricate network of interaction living creatures, with their own characteristics, strategy and experience, forming a dynamic system... Capturing this beauty is one challenge of this modelling PhD.\\
I must now give you an overview on the mountain grasslands}


%\section{Photograph of mountain grasslands}
The idea is to go from context, to services to ecology. At the entd of this part, it is obvious that ecosystem services can be derived from traits and that we need tools for prediction of MG dynamics.

\section{Geography, climate and managements: les drivers}


\section{Mountain grasslands under climate change}

\section{Mountain grasslands, source of services}

\section{Let's talk about traits} % might not be at the right place here
response trait and effect traits (?) <- you need to talk about this to better introduce the shift from there to a deep/low level traits to composite traits (SLA if related to a certain resource use strategy, is also a composite trait (Nitrogen, light and water all affect the SLA value). View such traits as mono dimensional is dangerous as it simplify a lot (and I might have fallen within this trap). Having low level traits that define the overall strategy (and not how to achieve it) should help to have a more systemic view (good term here ? view of the system with all its parts) and better understand the rules that drive the development. But they also imply the need for new mechanisms to link such traits and strategies to actual, measurable traits that define the phenotype as we see it (at the level of interaction or services profides). \\
Opportunity to emphasis the fact (a simple scheme should help here) that apparent traits are usefull (and measurable) to define the current effect on the environment (and so interactions between plants) and on provided services, but they are difficult to use to predict the dynamic of individuals and communities (precisely because they are changing and composit traits that respond to the environment).



%_________________________________________________________________________________
\chapter{Modelling ecological systems}
The message here should be that: we know how to model vegetation systems, but we need finer resolution and bigger scale -> generic framework and individual response.

\section{Models as understanding and testing tools}

\begin{quote}
"Physicien de la biologie"
\end{quote}
Justify the modelling approach - what's a model ? simplification of reality\\
Long subject refer to models in ecosystem sciences. Different classes of model, and different objectives. Mechanistic models: understanding and testing hypothesis.\\
Model as understanding tools: how does modelling help us understanding the system we are modelling.\\
The need for mechanistic model and emergent properties of models. Process-based models vs statistical model (what happen outside the data (example of flickering tails of regression models), similar to bayesian approach, the model is constrained by our understanding of processes.) \\

\section{Modelling plant communities}

\subsection{Different levels of modelling}

\subsection{Processes}

\subsection{Agent-based models}


\section{Modelling coexistence}
Message here ?

\subsection{What is diversity?}
\begin{figure}
\includegraphics[scale=1]{./Introduction/graphics/plankton.jpg}
\end{figure}

\subsection{The concept of niche}


\subsection{Coexistence mechanisms}

%_________________________________________________________________________________
\chapter{Phenotypic plasticity of organisms}
Message here ?

\section{Stability and plasticity}

\section{Costs and limits of plasticity}


%__________________________________________________________________________________
\chapter*{Scientific questions}
How to model vegetation system with higher resolution at bigger scale?\\
How does plasticity work in plants?\\
Effect of plasticity of plant interactions?\\
Effect of plasticity on resistance/resilience to climatic events?\\
Effect of this mechanism on overall services provision?




