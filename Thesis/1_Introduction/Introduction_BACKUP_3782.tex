% #######################################################################################
\chapter{Understanding community dynamics: drivers and theories of coexistence}

% _______________________________________________________________________________________
\section{The mystery of coexistence}

Other things being equal hypothesis (in models at least) does not allow the full diversity to emerge.\\

\textbf{One mechanisms alone seams to not be enough to explain fantastic diversity explain in natural ecosystems. However there are multiple theoretical mechanisms that support species diversity and that should taken into account in community models: diversity of resources, spatial and temporal variability, frequency dependent effects, etc...}

% _______________________________________________________________________________________
\section{Abiotic and biotic factors: from potential to realised niche}

Abiotic drivers main tnhing at global scale... Then interactions and competition.

\textbf{The concept of ecological niche serves as a great tool for theoretical research on coexistence. It encompass in a convenient way both abiotic and biotic drivers of one species distribution. The Hutchitonian niche also captures the multidimensionality of persistence and reproduction.}
% _______________________________________________________________________________________
\section{The complexity of coexistence}

\textbf{From the multiple first attends to explain coexistence with one particular mechanism, scientific community realised that indeed multiple mechanisms are at work to make species diversity in ecological community. This multiplicity highlight the need for unifying framework able to cover this diversity of mechanisms and dimensions.}

<<<<<<< HEAD
<<<<<<< HEAD
\section{Diversity and coexistence mechanisms}
Why interested in diversity? precious, main objective in conservation, plus services. Diaz 2001 , Hulot 2000\\
 Why coexistence mechanisms? Mechanism at plant level that allow diversity, understanding these will help us predict changes in diversity.
=======
Diversity of natural systems have long been a subject of admiration but also a mystery to the scientific community. The extraordinary multitude of species and individuals living at the same time, in the same space, is hard to reproduce and to explain. This is particularly true in plankton communities where the resources are limited to few nutrients (nitrate, phosphates) and light. Early coexistence theories, like the competitive exclusion principle that predicts that the number of coexisting species at equilibrium can at most be equal to the number of limiting resources, fail to explain such variety. This gap between empirical observation of high diversity in different systems and at different scales, and the lack of theoretical explanation was called the \textit{plankton paradox}. This paradox has now received multiples answers \textbf{NEED REFS}\sidenote{discussed in later in subsection \ref{ssec:div-mech}.}, but diversity and coexistence mechanisms are still investigated \cite{falster_plant:_2016}. This illustrates our will to understand these mechanisms, but why are we still struggling with questions that animated ecologist decades ago? and why are we still interested by these questions? It is hard to answer the first interrogation, but the diversity of the interactions within such systems and the diversity and variability of external drivers shaping them are the main factors. That also explains partly the second question, and why scientists explore the genetic diversity of gut bacteria, or the phylogenetic diversity of phytoplankton in lakes, or the diversity of plants species from tropical forest of Brazil to snowy slopes of Alps. But besides the curiosity of scientists, studying the machinery behind the functioning the natural communities is essential if we want to understand and predict how they can evolve under the pressure of changing drivers, and how they can be managed. The following paragraphs attempt to explain the value of diversity, and so why we have to predict and manage it at best, and where is our current understanding of underlying mechanisms.\\

\indent I may have been a bit far. Recentrate around mountain grasslands
>>>>>>> 7ed054d5706226acc1c6258f406795c654df3bda
=======
% #######################################################################################
\chapter{Considering strategies and functional traits}
% _______________________________________________________________________________________
\section{The continuity of functional ecology}

\textbf{The paradigm shift toward functional ecology allowed the shift from discrete to continuous representation of species. This change make easier the representation and study of plant communities, especially along conditions or management gradient. Despite the advantages of functional traits, close comparisons and links with theoretical approaches should be used carefully, and underlying assumptions should be interrogated.}
>>>>>>> f2dcea767a1fd1400d61e1fb10eef4ae3a641203

% _______________________________________________________________________________________
\section{How trade-offs make strategy space}

Diversity of mech: diveristy of strategies. more or less independent.\\

\textbf{The multiplicity of mechanisms in these systems lead to similar diversity in plant strategies. These strategies are captured in a strategy space drawn by independent trade-offs. Such trade-offs can be captured and embodied thanks to functional traits.}


% _______________________________________________________________________________________
\section{Modelling diverse plant community}

\textbf{The use of strategy spaces in models allow the representation of high diversity in a common plant functioning framework requiring limited number of parameters. Such approaches are very useful to follow the dynamics of communities in a mechanistic framework. }

% #######################################################################################
\chapter{The importance of intra-specific variability}

% _______________________________________________________________________________________
\section{Intra-specific variations change the rules}

More interest in trait distribution, variability and diversity. $\rightarrow$ Get to look at intra-specific variability.\\
Jung: not always in the same way\\
Wellstein: 

\textbf{The intra-specific variability has been observed to be both an important part of community functional diversity, but also a way the community respond to changes in conditions. In addition to the empirical evidence of this importance, theoretical approaches support contrasting effects of such variations on coexistence mechanisms, evolutionary processes and community responses to climate event or invasion. It is crucial to disentangle different sources of intra-specific variability in order to their understand potential effect on ecosystem dynamics.}

% _______________________________________________________________________________________
\section{Phenotypic plasticity: a specific case}

%Confusion, phenotypic plasticity is particular phenomenon, driven ...\\
\textbf{•}


% _______________________________________________________________________________________
\section{Beyond the mean and the bell-shape: for more mechanisms in intra-specific variability}
The same way the neutral theory is simplifying and brings little understanding to underlying processes and relies on strong hypothesis, considering intra-specificity as a purely random mechanism is insufficient.\\
Bell shape do not appear in altitude gradient...\\
Strong theoretical hypothesis\\
Asymmetric and symmetric competition\\

\textbf{As ecology shifted from species to traits syndroms, it seems that it needs to go from syndroms to distributions and drivers. The complexity of living communities requires to go further down and consider the individual scale. This is made possible by the acumulation of more and more numerous and detailed data, the improvement of statistical and simulation tools. }

% _______________________________________________________________________________________
\section{For an integrative framework of plant strategy and phenotypic plasticity}

Bradshaw?
Dewitt\\

\textbf{New simulations tools for understanding community dynamics should try to both include multiple coexistence mechanisms and plant strategies, and focus on individual level mechanisms of competition, growth and survival. This can only be achieved an a constraint high dimensional strategy space based on physical and biological trade-offs. Individual level modelling allows the integration of multiple sources of intra-specific variability: genetic diversity and phenotypic plasticity. Phenotypic plasticity being driven by the perception of environment, it cannot be simply described by normal random distribution and should receive more attention. This focus is particularly important considering both the lack of understanding of this phenomena and the consequences for plant communities.  }

% _______________________________________________________________________________________

%\chapter{A new interface to explore} %##################################################
%
%
%\subsection{Vegetation community}
%
%\subsection{Basis of coexistence}
%
%\paragraph{Niche theory}
%
%%_______________________________________________________________________________________
%\section{On strategy and traits}
%
%\subsection{From species to traits}
%
%\subsection{Traits and strategy space}
%
%%_______________________________________________________________________________________
%\section{What is phenotypic plasticity}
%
%\subsection{The importance of intra-specific variability}
%
%\paragraph{Mean traits are not enough}
%
%\paragraph{The source of the variation}
%
%\paragraph{Different impacts on mechanisms}
%
%\subsection{Phenotipyc plasticity: a form of intraspecific variation}
%
%\paragraph{From genotype to phenotype}
%
%\paragraph{New references}
%
%\subsection{Extent of plasticity}
%
%\paragraph{Advantage}
%
%\paragraph{Costs}
%
%\paragraph{Limits}
%
%%_______________________________________________________________________________________
%\chapter{Strategy space and community modelling} %####################################
%
%
%
%%_______________________________________________________________________________________
%\chapter{Phenotypic plasticity and trait variations}   %##############################
%
%%_______________________________________________________________________________________
%\chapter{Community dynamics and the role of intra-specific variability} %#############
%

%
%\chapter{On coexistence and diversity}
%
%\section{Diversity and coexistence mechanisms}\label{sec:coexistence}
%
%Diversity of natural systems have long been a subject of admiration but also a mystery to the scientific community. The extraordinary multitude of species and individuals living at the same time, in the same space, is hard to reproduce and to explain. This is particularly true in plankton communities where the resources are limited to few nutrients (nitrate, phosphates) and light. Early coexistence theories, like the competitive exclusion principle that predicts that the number of coexisting species at equilibrium can at most be equal to the number of limiting resources, fail to explain such variety. This gap between empirical observation of high diversity in different systems and at different scales, and the lack of theoretical explanation was called the \textit{plankton paradox}. This paradox has now received multiples answers \textbf{NEED REFS}\sidenote{discussed in later in subsection \ref{ssec:div-mech}.}, but diversity and coexistence mechanisms are still investigated \cite{falster_plant:_2016}. This illustrates our will to understand these mechanisms, but why are we still struggling with questions that animated ecologist decades ago? and why are we still interested by these questions? It is hard to answer the first interrogation, but the diversity of the interactions within such systems and the diversity and variability of external drivers shaping them are the main factors. That also explains partly the second question, and why scientists explore the genetic diversity of gut bacteria, or the phylogenetic diversity of phytoplankton in lakes, or the diversity of plants species from tropical forest of Brazil to snowy slopes of Alps. But besides the curiosity of scientists, studying the machinery behind the functioning the natural communities is essential if we want to understand and predict how they can evolve under the pressure of changing drivers, and how they can be managed. The following paragraphs attempt to explain the value of diversity, and so why we have to predict and manage it at best, and where is our current understanding of underlying mechanisms.\\
%
%\indent I may have been a bit far. Recentrate around mountain grasslands
%
%\subsection{Effects of diversity}
%Conservation\\
%productivity\\
%resistance ?\\
%Ecosystem services and complementarity\\
%
%\section{Mechanisms for coexistence, trade-offs and strategy spaces}\label{ssec:div-mech}
%main theories: niche, neutral, individual based. -> scale and dimension dependant.\\
%chesson \cite{chesson_mechanisms_2000}\\
%Spatial and temporal variability\\
%trade-off, strategy space, and variability.\\
%in the end it's rarely direct interaction but capacity to respond to stress and interect interaction through resource pools.
%
%
%\section{About trade-off}
%chemical physical trade-off vs ecological trade-off.
%
%
%\section{Strategy spaces}
%
%
%\chapter{Intra-specific diversity and plasticity}
%\label{sec:intraspe}
%
%
%\section{Community dynamics: from individuals to group dynamics}
%\textbf{Need to  highligth how community dynamics emerge from individual response and interactions.}
%
%\section{Intra-specific variability}
%frame of reference: deep traits vs shallow traits. definition of functional trait.\\
%source of intra specific variability: genetic vs ontogeny vs plasticity (epigen) \\
%effect on niche and interactions: effect on coexistence\\
%-> plasticity a special form of ISV
%
%\section{Understanding phenotypic plasticity}
%
%%what is it, how it works or doesn't
%
%Bradshaw, sultan
%
%adaptive intraspecific variation\\
%cost and limits van kleunen, Dewitt and sultan \\
%effect on coexistence and community\\
%
%\begin{fullwidth}
%\begin{tcolorbox}[title=Molecular basis of phenotypic plasticity] %Toutes les options définies dans le préambule peuvent être définies aussi ici. J’ai juste gardé la possibilité de changer le titre de la box dans ma thèse
%Phenotypic plasticity lies both in the perception of external conditions through sensor organ and signaling pathways (auxin pathway, root stones for gravity ...), and the integration of this information to alter the development plan. This integration must be coordinated at the scale of the plant according to rules or objectives, question partly explore in this work, but ultimately is applied at the cell levels.\\
%\indent Because of the complexity and our partial understanding of these mechanisms, we will not attempt to model them. However I hope that this little overview of molecular mechanisms at the scale of the cell will give the reader an idea of the processes behind the abstract concepts used in this manuscript.\\
%
%BLABLABLA and figure\\
%
%The diversity of mechanisms and scales (both spatial and temporal) these processes can act inside of plant gives an idea of the diversity of strategies a plant can deploy to face changes of its environment. Considering this complexity, only a small fraction can be explored in such model as \model, but hopefully it will help make progress in our understanding of the role of these molecular mechanisms at the scale of the community.
%\end{tcolorbox}
%\end{fullwidth}
%
%\chapter{Niche, competition and coexistence with intraspecific variability}
%
%\textbf{Go beyond bell-shaped niche and symmetric competition. Trait analysis and mechanistic approaches defend more complex theories and complexity. Need tools integrating flexibility and complexity. Science is measure the relative balance between different effects/mehcanisms. Cannot be simplified to one simple mechanisms, but look when (what conditions) is more important than the other, how one is closer to real system, what properties this has.}
% 
%
%\chapter{Existing modelling approaches}
%\section{Global change effect on vegetation community}
%
%Message: modelling coexistence is a challenge because 1) do not know/understand all mechanisms, 2) challenging to incorporate enough mechanisms, 3) costly computation and data wise. -> need for more generic and complete (multiple mechanisms approaches.\\
%
%DGVMs\\
%IBMs
%
%
%\section{Modelling vegetation - traits and strategies}
%traits \& strategies\\
%existing models: a gap to fill\\
%coexistence processes
%
%\section{Modelling phenotypic plasticity}
%Reaction norms\\
%Source sink models\\
%Functional-Structural plant models FSPMs ? vos 2009\\
%Functional equilibrium. Somehow similar to the source sink in its philosophy, it allows optimisation of phenotype for multiple resources. \\



\paragraph{Individual based and spatially explicit}
\textit{This might not be at the right place}\\
Need to be individual based if we want individual variations with plasticity (not necessary when genetic diversity). \\
Need for spatially explicit model because avoid tragedy of the common \cite{farrior_competitive_2014}.

%
%
%%_________________________________________________________________________________
%\chapter{Mountain grasslands}
%\fwnewthought{Mountain grasslands have an unique beauty drawn by the diversity of flowers colours, the strong contrast between the luxurious green vegetation and the roughness of the naked rocks, and the feeling that living in such places, at the edge of living conditions, is a fight worth fighting. I could illustrate this beauty with thousands pictures and words, and it would certainly convince you that these ecosystems worth spending time studying them to better understand and protect them. I could also describe their role in the economy of alpine regions, currently subject of great modification, and greater to come. But you would not see these rich systems as I see them: as an intricate network of interaction living creatures, with their own characteristics, strategy and experience, forming a dynamic system... Capturing this beauty is one challenge of this modelling PhD.\\
%I must now give you an overview on the mountain grasslands}
%
%
%%\section{Photograph of mountain grasslands}
%The idea is to go from context, to services to ecology. At the entd of this part, it is obvious that ecosystem services can be derived from traits and that we need tools for prediction of MG dynamics.
%
%\section{Geography, climate and managements: les drivers}
%
%
%\section{Mountain grasslands under climate change}
%
%\section{Mountain grasslands, source of services}
%
%\section{Let's talk about traits} % might not be at the right place here
%response trait and effect traits (?) <- you need to talk about this to better introduce the shift from there to a deep/low level traits to composite traits (SLA if related to a certain resource use strategy, is also a composite trait (Nitrogen, light and water all affect the SLA value). View such traits as mono dimensional is dangerous as it simplify a lot (and I might have fallen within this trap). Having low level traits that define the overall strategy (and not how to achieve it) should help to have a more systemic view (good term here ? view of the system with all its parts) and better understand the rules that drive the development. But they also imply the need for new mechanisms to link such traits and strategies to actual, measurable traits that define the phenotype as we see it (at the level of interaction or services profides). \\
%Opportunity to emphasis the fact (a simple scheme should help here) that apparent traits are usefull (and measurable) to define the current effect on the environment (and so interactions between plants) and on provided services, but they are difficult to use to predict the dynamic of individuals and communities (precisely because they are changing and composit traits that respond to the environment).
%
%
%
%%_________________________________________________________________________________
%\chapter{Modelling ecological systems}
%The message here should be that: we know how to model vegetation systems, but we need finer resolution (from species and com, to traits, to individual responses) and bigger (higher number of species) scale -> generic framework and individual response.\\
%Based on two particular similar models: taubert, and Lohier.
%
%\section{Models as understanding and testing tools}
%
%\begin{quote}
%"Physicien de la biologie"
%\end{quote}
%Justify the modelling approach - what's a model ? simplification of reality\\
%Long subject refer to models in ecosystem sciences. Different classes of model, and different objectives. Mechanistic models: understanding and testing hypothesis.\\
%Model as understanding tools: how does modelling help us understanding the system we are modelling.\\
%The need for mechanistic model and emergent properties of models. Process-based models vs statistical model (what happen outside the data (example of flickering tails of regression models), similar to bayesian approach, the model is constrained by our understanding of processes.) \\
%mechanistic models: risks of lack of mechanisms, complex calibration, lot of parameters. Statistical model have the advantage of parcymony: minimum number of parameters to reproduce a pattern.
%
%\section{Modelling plant communities}
%
%\subsection{Different levels of modelling: from communities to individuals}
%community approaches, CWM to importance of individuals.\\
%
%
%\subsection{Processes}
%\paragraph{Test pragraph title - now what happen if it lays on multiples lines} This is a paragraphe \lipsum[2]
%
%\marginnote{\textbf{some notes}}
%\marginnote{\lipsum[1]}
%\sidenote{a short sidenote}
%
%\subsection{Agent-based models}
%
%Review of existing models (grasslands and forest)\\
%Comparison of two existing models\\
%How to build aroung/from that.\\
%
%
%\section{Modelling coexistence}
%Message: modelling coexistence is a challenge because 1) do not know/understand all mechanisms, 2) challenging to incorporate enough mechanisms, 3) costly computation and data wise. -> need for more generic and complete (multiple mechanisms approaches.
%
%\subsection{What is diversity and why model it?}
%
%On why model coxistence: better understanding of mechanisms, tool to evaluate and predict changes in coexistence.
%
%\begin{figure}
%\includegraphics[scale=1]{./1_Introduction/graphics/plankton.jpg}
%\end{figure}
%
%\subsection{The concept of niche}
%
%
%\subsection{Coexistence mechanisms}
%
%\section{Breaking the resolution-specificity trade-off}
%use of generic species\\
%still a trade-off with scale.
%
%%_________________________________________________________________________________
%\chapter{Phenotypic plasticity of organisms}
%Message here ?
%
%\section{Stability and plasticity}
%
%\section{Costs and limits of plasticity}
%
%\section{Plasticity and coexistence}
%
%
%%__________________________________________________________________________________
%\chapter*{Scientific questions}
%How to model vegetation system with higher resolution at bigger scale?\\
%How does plasticity work in plants?\\
%Effect of plasticity of plant interactions (and coexistence)?\\
%Effect of plasticity on resistance/resilience to climatic events?\\
%Effect of this mechanism on overall services provision?




