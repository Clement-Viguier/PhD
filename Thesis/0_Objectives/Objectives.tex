
%\chapter{Objectives}

\chapter{Context}

\section{Global change: how to describe the future of alpine ecosystems?}

\subsection{The value of ecosystems: from properties to services}

\paragraph{A new logic}
Everyone has a particular relationship with nature. The vision we put behind this word depends on the way we experienced nature, it can be temperate or tropical forest, mountain rivers or cliffs on the ocean littoral, bird songs or wind between stones. Anybody that shares one of these visions, I am sure wants to preserve natural systems. But facing this emotional perception and inner desire to see these ecosystems be preserved, other forces pushes in other directions. The reduction of biodiversity is increasing at dangerous rates, the deforestation threaten the largest forest systems, insects are less and less presents and animals are repelled to fragmented and diminishing habitats. Other logics than emotional attachment and will to protect impact all natural systems around the world. To be protected, the natural systems needed a way to be integrated in these logics, and the notion of \textemph{ecosystem services} was developed by \cite{costanza_value_1997}. This notion encompass all benefits human extract from ecosystems. It enables a categorisation of services and their quantification (that can go to the monetisation), and therefore allow them to be taken into consideration in global logic of capital, investment and value.


\paragraph{Services}
The notion of ecosystem services aims to capture the value of ecosystem, but what is this value?

If ones could be tempted to answer that the value of an ecosystem cannot be measures, it is clear that all ecosystems do not benefit to human in the same way. Face to the diversity of ecosystems and services they provide, we can try to develop a short answer for the object of study to this document: mountain grasslands.

\textemph{Mountain grasslands} designs in this document all grasslands, below and above the treeline, that have short growing seasons delimited by snow covered periods and experience high variations in temperature and water availability. This term is intentionally generic as the scope of this work is relatively broad and theoretical.

Mountain grasslands provide numerous services, that can be divided in multiple categories such as provision, cultural and regulating services. Provision services are related to the quantity and quality of primary resources the grasslands provide. Fodder production and quality are the main measures of provision services. Other services can be included in this category: diversity of flowers and phenology for flower production for instance. Productivity is also interesting to assess carbon capture, a regulating service. Soil nutrient availability and water filtering are other regulating services impacted by the identity and diversity of species populating mountain grasslands. Finally, cultural services, related to tourism activity and landscape appeal are also related to grasslands species diversity.


In case of terrestrial ecosystems, vegetation cover is often central because of: it role of primary production, and the fact that vegetation community informs a lot on the properties of the abiotic and biotic conditions. Moreover, a most of studies on services from terrestrial ecosystem are interested in plants and soil invertebrate \cite{de_bello_towards_2010}, revealing the importance of vegetation in the provision of ecosystem services. In addition, in alpine habitats plant communities are susceptible to be the first impacted by global change because they cannot escape changes in conditions and are the target of management practices linked to fodder productions. All these arguments support the interest of studying the vegetation dynamics for the assessment of ecosystem services.

\begin{figure*}
\includegraphics{./1_Introduction/graphics/alpine_distribution.jpeg}
\caption{Distribution of alpine habitats. Alpine habitats shelter unique and rich ecosystems providing numerous services to human populations. Climate change and mutations of land-use practices threaten these dispersed and fragile habitats.}
\end{figure*}

\paragraph{Properties}
The ecosystem services are tightly related to the \textemph{ecosystem properties} (as illustrated in figures \ref{fig:properties})\parencite{lavorel_predicting_2002, diaz_incorporating_2007} that can be extracted from the description of the grassland communities. Ecosystem properties are features of the community that characterise it and arise from the characteristics of all parts of  the system or how they combine. The main properties of a plant community are capture in the following concepts:
\begin{itemize}
\item \textemph{identity}: the identity of the community refers to the dominant species (or directly its characteristics) of the community that transfers its traits to the whole community. It can also refer to mean traits (with community weighted mean measures) of a community. In this document, identity will often be used to talk about the resource use strategy (more or less exploitative). While this notion can encompass multiple traits and measures, it is practical to use one term to identify components of the community description that can be attributed to a species\sidenote{in opposition to variables that are related to a system, \textit{e.g.} diversity cannot be expressed for a species alone};
\item \textemph{diversity}: diversity plays a large role in the provision of multiple services, and is related to other properties of the community. Diversity can be expressed in term of species richness or functional diversity\sidenote{each measure depending on the functional space that is considered}, and by a wide range of indexes that are not discussed here. Despite a lot of nuances between these notions, they are often tightly correlated and diversity will be discussed in term of number of species or functional volume in the rest of this document.
\item \textemph{productivity}: productivity captures the capacity of the system to produce organic matter in a given timespan. It is a ambiguous term as it can refer to the abiotic environment, to a species or a community property or even to a service. I will try to limit its use to the species or community relative vegetative biomass in a given condition.
\end{itemize}

%-----------------------------------------------------------------------------------------
%\subsection{From community description to ecosystem services: the facets of the community}

%Ecosystem services are various. Some of them can be easily assess (e.g. fooder production and quality), while others are more subjective (cultural or recreational services) or hard to measure (carbon sequestration, water purification etc...). But all of them rely on a good description of the system, even though this description do not have to be complete as certain aspect of an ecosystem might not be relevant to all provided services.


Linking ecosystem services to ecosystem properties is essential both for the understanding of processes controlling these services, and for an easier quantification of such services. This is particularly important for the prediction of services levels to plan management practices in the context of global change. Some ecosystem services are here linked to the main community properties as illustrated in figure  \ref{fig:properties}.
% The question of the description and prediction of plant communities properties and dynamics will be addressed more in details in the following sections of this chapter, but it is important to establish the main components of a vegetation community link with provided services.


\paragraph{Diversity}


\paragraph{Identity}

\paragraph{Productivity}
Mountain grasslands provide numerous ecosystem services 

ecosystem services depends on abiotic, but also biotic factors and properties. 

%
%%-----------------------------------------------------------------------------------------
%\subsection{The facets of plant communities}

%

%The assessment of ecosystem services relies on a detailed characterisation of the community structure and properties. The knowledge of species characteristics and relative abundance allows the computation of summary variables that characterise the plant community. Long history of plant study and description gives us good knowledge of benefit provided by specific species. 

This structure is defined by the relative abundance of the different species of the community. Multiple drivers affect the relative abundance of a given species, from abiotic filtering processes to biotic interactions. 

Need of mechanisms to produce dynamics and give properties.

\textbf{The complexity of plant community dynamics requires mechanistic approaches to understand and predict system properties in new, extreme, and variable conditions. }


\textbf{The evaluation of ecosystem services relies on a precise description of the ecosystem abiotic and biotic properties. The plant community is the most dynamic and complex driver of ecosystem services, but direct links can be drawn between the fine description of the community and the ecosystem services. Understanding and prediction the main variables dynamics that capture those links is necessary to efficiently predict changes in ecosystem services levels.}
\textbf{Plant communities are complex interconnected systems. In order to evaluate ecosystem services, they can be summarised by three main types of variables that capture different dimensions of such systems: the diversity, the productivity and the identity. These dimensions can be studied independently or jointly and give different information on secondary properties and provided services. But grassland communities are natural systems driven by environmental variables, and these drivers are changing leading to changes in services.}



\subsection{Global change: what changes and what consequences}

Mountain grasslands are maintained by strong climatic constraints that limit growth rate and lifeforms  \parencite{koorner_alpine_2003}, but also frequent grazing or cutting perturbation regimes that strongly limit the growth woody species and favour low stature species or rapid growth herbs \parencite{diaz_plant_2007}. But these drivers are changing at alarming rates and mountain grasslands are suspected to be very vulnerable \parencite{engler_21st_2011} due to higher variations in water availability regimes and specific warming processes \parencite{mountain_research_initiative_edw_working_group_elevation-dependent_2015}, stronger isolation (island effect due to rise in temperature) and reduction of the grazing pressure.

\paragraph{Climate change}
Changes:
Rising temperatures due to anthropogenic greenhouse gases has a strong effect on mountain climate. 

Consequences: contrasting, depends on the factor: co2 or drought

\paragraph{Land-use mutations}

trade-off lavorel and \parencite{schirpke_multiple_2012}

management change the position along these trade-off

climate also change things



 
%
%\section{Community dynamics: complexity emerging from parts and the role of phenotypic plasticity}
%title too vague to bring meaning, should put both parts together.
\section{Models: a solution to understand and predict complexity}

\subsection{The need for mechanistic models}

\paragraph{A new world}
outside what's known, extrapolations and experimentations

The combined effect of land-use mutations and climate changes will lead to environmental conditions never experienced by such systems. Predicting the future in new conditions implies extrapolating multiple effects not tested in combination: with cumulative effects and potential synergies (carbon dioxide increase and grazing abandonment) or effects balancing each others (grazing abandonment and higher frequency drought events).


 \paragraph{Complexity}
 this title is not helpful -
 
 combined effects
 
 community responses: different processes (recruitment, growth, plasticity etc...) \& levels (indiv, pop, metacommunity)
 

 In addition to complexity of combined effects of global change drivers, complexity is inherited from the complexity of the community dynamics. Interacting species may change response of the system, and should be better taken into accounts \parencite{gilman_framework_2010}. To answer this challenge, large scale experiments are conduced such as Cedar Creek experiment in the United-States, or JENA experiment in Germany. These experiments give high value experimental data for various conditions and a variety of species, where interactions can be studied as well as management effects.
 Transplant experiments are also conduced to investigate the effects of temperature rise on the productivity, diversity and structure of the community \cite{scheepens_genotypic_2010}(Need more references) Showing increase in productivity and dicrease in diversity, as well as a shift toward more acquisitive species \parencite{debouk_functional_2015}.
 
 Observed effects: jung, transplant, effect on diversity and productivity.
 
contrasting effects as function of elevation: change in identity (abundance) and increased diversity in low altitude, but decrease in diversity in high altitude \parencite{rosbakh_elevation_2014}
 
 But, temporal effects, history (that guy from ecoveg talk) hysteresis effect, metapop and invasion effects, balance between intra-specific and einter-sp responses...
 
 Modelling approaches   
 
 
limits of empirical studies: \parencite{merila_climate_2014}

 
 \parencite{schirpke_multiple_2012}

 The increasing variability in those conditions 
 
 and uncertainty that would require multiple experiments. Models allow to explore multiple scenarios.

\subsection{The limit of classic patterns}

niche vs process: stronger effects because no plasticity or local adaptation \cite{morin_comparing_2009}
\subsection{The rise of individual-based approaches}

LINGRA-CC \cite{rodriguez_lingra-cc:_1999} to test gc effect on productivity : higher productivity allowing shorter intervals between cutting

Maire

Lohier: vegetative phase, coexistence and ontogeny... 

Taubert: diversity productivity 


\subsection{When phenotypic plasticity makes things complicated}

plasticity change response \cite{morin_comparing_2009}

phenotypic changes in competition intensity that increase negative effect \parencite{hanel_phenotypic_2015}

plus ignored effects of intra-specific variations: additional level of response: amplification or mitigations, driver dependencies?

\subsection{Gaps to fill}

A wide range of models have been developed to better understand biological processes involved in plant growth and population dynamics, from organ-based models to functional types approaches.

As the scale increases, the resolution diminishes and the verticality of processes is rarely taken into consideration. It is not a problem in stable conditions, as the lower levels are implicitely integrated in the grain of larger processes (like the leaf gaz exchanges regulation processes are ignored at the scale of the population). But 2 things:
(1) ok to not explicitly represent if know and considered within a broader mech (translated into assumptions: \textit{e.g.}: assumption that stomata regulation), it is not the case of phenotypic plasticity as it is not considered in basic assumptions made. Plus, it depends on the scale, but daily growth require plasticity, period.
(2) they may greatly change plant and community behaviour in changing conditon/environment.
 
scales and processes (climate, management etc...)
put the resoure in the center (fate-hd)

process and mechanisms
\parencite{berger_competition_2008}: effect on local env., adaptive beh, below-ground.
partly filled (maire and Lohier).

but lack of species diversity and genericity. 

%
%
%
%\section{Global change and community dynamics in alpine grasslands}
%\begin{figure*}
%\includegraphics{./1_Introduction/graphics/alpine_distribution.jpeg}
%\caption{Distribution of alpine habitats}
%\end{figure*}
%
%Climate change is probably the greatest challenge the humanity has to face this century. Expected drastic changes in both average climatic conditions and punctual climatic event frequencies and intensities will, and already have, an impact all around the globe on multiple aspects of our lives. From agricultural and economic, to social and political, but also scientific and technical, the problems for human societies are numerous and multidimensional.\\ 
%Need to better understand and predict natural systems. Mountain grasslands are susceptible to be greatly impacted (even if certain think they might not). And in new ways as the rising temperature will certainly lead to migration to higher altitude, increasing the island characteristic of alpine habitats and reducing links between communities, and at the same time increasing the opportunity of invasion by lower altitude higher temperature species.\\
%\indent Detail a bit the characteristic of mountain grasslands, (snow, islands, grazing) the effect on species (snow-bed species, link to meta-community, diversity, species adaptation to frost etc... and how global change may affect that.\\
%\indent Because of that mountain grasslands are rich in species, but also vulnerable, that is why in parallel of predicting climate change, we also need to understand ecological mechanisms under this diversity and how they can be affected by global change \sidenote{section \ref{sec:coexistence}}. A key part in community diversity and in adaptation of communities also lies in the diversity and adaptation of individuals, so we are interested in intra-specific diversity and phenotypic plasticity \sidenote{section\ref{sec:intraspe}}.
%
%
%%Take home message ####################################
%\textbf{There is a need for new tools to predict the response of ecosystems to new climate conditions and management scenarios. These tools should integrate the complexity of such system and the mechanisms underlying the dynamic responses of these communities.}
%
%\section{Empirical results, trait approaches and need for a new kind of model in grasslands}
%
%\subsection{On trait-based approaches}
%
%Holy Graal of ecology\\
%Lavorel, Kraft, Kunstler
%
%\subsection{The importance of intra-specific variability}
%
%Jung
%Leps
%Albert
%Kichenin
%Lavorel (hypothesis of traits bell shape)
%Violle ...
%
%%Take home message ####################################
%\textbf{Trait approaches allow for generalisation and more direct link with processes and services. However they ignore variations and processes at lower levels than the species that are of critical importance for the understanding of community dynamics. A mechanistic approach integrating processes at the individual level and rich community complexity are needed.}
%
%\section{Close a gap in grassland modelling}
%
%%Take home message ####################################
%\textbf{Generalizing models for forest ecosystems and complex individual level models for grasslands coexist, but there is a need for a generalizing model at individual scale for grassland communities.}
%
%\section{Effect of phenotypic plasticity on coexistence and community dynamics}
%
%%Take home message ####################################
%\textbf{Despite empirical and theoretical work, the effects of intra-specific variability and plasticity on community dynamics are not fully disentangled. Understanding the effect of individual variations on plant community is crucial and may greatly alter how we envision the future of these ecosystems.}
%
%%_________________________________________________________________________________
%

\chapter{Aims, Objectives and Overview}


\section{Aims: understanding and prediction}

Global change is probably the biggest challenge humanity has to face at the beginning of this millennium. But while action is needed, it requires understanding, and the multiplicity of environmental drivers impacted by global change, whose effects can synergise or balance themselves, in addition to complex structure and dynamics of natural systems make this understanding hard to build and to summarise.

To go beyond traditional pattern-driven ecology and overcome the difficulty of combined causes leading intricate effects, mechanistic approaches should be priviledge. 

The functioning of individuals living in these communities and the dynamics of the resources should be at the core of the new approaches to better understand the trajectories of the ecosystems.

\textemph{Ecosystem} both living and non living components of a systems binded together by interactions. 
%
%Functioning
%Diversity of : drivers, mechanisms, species and strategies
%Flexibility: structure: genericity, experiments, plasticity

\section{Objectives: a new agent-based model for plant community dynamics} % the why
Traditional empirical approaches of observation and controlled experiments provided valuable information on the functioning of these systems. However, they lack power to understand intricate systems and predict their dynamics, especially in case of uncertain scenarios. 

Modelling approaches must be used to build understanding and predictions of natural ecosystems dynamics driven by changing environmental drivers. These models should include the diversity of drivers as well as the diversity and the intrinsic complexity of these systems.

In order to be compensate long development time and to extend the reach of experimental approaches, models should try to keep generic in structure and flexibility in use, while being specialised thanks to parameters or simple equation changes.

\subsection{Generic framework for multi-species and plastic plant modelling} % the how

In the context of mountain grasslands, showing unique levels of diversity despite strong environmental drivers, species diversity cannot be ignored to predict the response of the community. This diversity must be translated into plant functioning differences leading to diverse niches and possible response. In addition to species level dynamics driven by these differences, intra-specific responses cannot be ignored, and a phenotypic plasticity mechanism is needed.



%trade-off that constrain inter and intra differences in the same way

\subsection{Effect of phenotypic plasticity on plant growth and community dynamics}

Intra-psecific variations are espectic to play an important role in the response of mountain grassland communities to global change. The effects of phenotypic plasticity and other source of variations must be disentangled. Explicit integration of phenotypic plasticity in a plant community model will help identify and understand these effects.

As multiple services derive from the main properties of the vegetation of mountain grasslands, it is crucial to establish how phenotypic plasticity specifically impact these properties. Because these properties depend both on properties of the individuals and the relative abundance and diversity of species, effects on processes at both individual and community scales must be investigated.


\section{Thesis overview}

The rest of this thesis is divided in five chapters. The following chapter \ref{part:literature}, in the form of a literature review, introduces the concepts and knowledges that support the approach developed in later chapters. The chapter \ref{part:model} develops the generic framework for plant functioning and phenotypic plasticity from the concepts established in chapter \ref{part:literature} and further extended. Chapters \ref{part:individuals} and \ref{part:community} present respectively individual and community scale results of simulations made with the developed model \model on the effects of phenotypic plasticity on main plant community properties. Finally, the final chapter discusses the outcomes of this work and present path to follow from the present conclusions. Extensions to develop on the model are also proposed.
