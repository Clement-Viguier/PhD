
%\chapter{The effect of phenotypic plasticity on plant community dynamics}
%Hypothesis on the cumulative effect on niche and interactions.
%
%\section{Individual resistance and resilience against drought events}
%Amplitude and length of the event :\\
%- severity effect reduced by lower tau ?\\
%- resistance versus resilience: H0: conservative strategy have higher resistance, H1 : low tau allows for re-equilibrium and increase resistance (low amplitude and long length. H2: high tau allow to avoid dead-end situation during short severe drought (high resilience)
%\section{Community response to drought event}
%coexistence effect vs resistance/resilience effect\\
%uniform vs heterogenous (plasticity wise) community response
%H1: 
%

\chapter{Community level calibration}

\section{Method}

\paragraph{Field calibration}
New random parameters sets (with no species specific parameters) for population dynamics and competition specific parameters (see table ...).
Sequence of around 60 year for each site. Parameters were selected by...


\paragraph{Field data}
Field data has been collected between years 201 .. and 201 by Claire Deleglise and al. ().

\paragraph{Weather data}
Weather data has be computed by the MeteoFrance model SAFRAN by ... using GPS coordinates and slope, azimuth and horizon computed from a "MNT". These parameters were also used by the model CROCUS to compute snow accumulation and melting. These high frequency data (resolution under 1h) have been average daily and used to compute input variables for \model . 

\section{Results}

\chapter{Plasticity: a winning strategy ?}

\chapter{Plasticity mechanism of coexistence}