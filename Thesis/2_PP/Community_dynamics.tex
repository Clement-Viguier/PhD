
%\chapter{The effect of phenotypic plasticity on plant community dynamics}
%Hypothesis on the cumulative effect on niche and interactions.
%
%\section{Individual resistance and resilience against drought events}
%Amplitude and length of the event :\\
%- severity effect reduced by lower tau ?\\
%- resistance versus resilience: H0: conservative strategy have higher resistance, H1 : low tau allows for re-equilibrium and increase resistance (low amplitude and long length. H2: high tau allow to avoid dead-end situation during short severe drought (high resilience)
%\section{Community response to drought event}
%coexistence effect vs resistance/resilience effect\\
%uniform vs heterogenous (plasticity wise) community response
%H1: 
%

\begin{fullwidth}
This second result chapter examines the effects of phenotypic plasticity at the scale of the community. Another parameter filtering processes is performed and described in the first section of this chapter. The second part focuses of the effects of plasticity of the main properties of the community. The impact of plasticity on species diversity is particularly investigated. This chapter gives a glimpse of the potential of the model to answer various questions around the role of intraspecific variations on diverse community properties.
\end{fullwidth}

\chapter{Community level simulations: non plastic community}


\section{Parameter filtering}
\subsection{Method}

%\paragraph{Field data}
%Field data has been collected between years 201 .. and 201 by Claire Deleglise and al. (). Not used here.

\paragraph{Weather data}
Weather data for the time period between 1959 and 2014 has be computed by the MeteoFrance model SAFRAN by ... using GPS coordinates, slope, azimuth and horizon computed from a digital elevation model. These parameters were also used by the model CROCUS to compute snow accumulation and snow melting. These high frequency data (resolution under 1h) have been averaged on a daily time-step and used to compute input variables for \model. The snow in particular defines the length of the growing season starting with the first snow melt of the year and finishing the day of the first snow fall of autumn or winter.

The simulated years above 2014 are randomly sampled form the existing dataset between 1995 and 2014.

\paragraph{Parameter filtering}
Community level parameter filtering is conduced for a new table of parameter sets. These parameter sets are ... from accepted parameters and joined with LHS random sampling for five community level parameters: seed germination density, drought mortality, ageing mortality, plasticity cost for environmental sensing and plasticity cost for trait changes (see chapter \ref{chapter:model-description} for details).

Few words on why plasticity cost parameters: time limits, distinguish the benefit of plasticity itself, not combined effect. Should have done simulations with no cost to have an idea of plasticity cost effect. 

The simulations run over 300 hundreds years for 6 sites described in table \ref{table:sites} on squares of ... square centimetres. The simulation is stopped and the parameter set rejected if no individual persist and the seedbank is empty. The seedrain is composed of seeds contained in the seedbank and seeds from the metacommunity. The total of seeds is defined by the seed germination density and the area simulated. The seeds from the simulated community represent up to 80 \% of the seedrain, less if the seed production is limiting. The first ... years are not taken into account in the filtering process to let the community settle.


\paragraph{Simulations}

\subsection{Results}

Simulations done. Need to illustrate the results.\\

\paragraph{Effect of parameters}
On stability and on diversity (functional and species)\\
Random forest approaches like sensitivity analysis at individual scale.


\section{Non plastic communities}
Trade-off, diversity, stability ...


\paragraph{Ecological trade-off ?}
Is there a selection of some parameters ? Are there ecological trade-off (resource use strategy and reproduction) emerging from the model ?

\chapter{Plasticity: impact on species fitness and diversity}

Plasticity in integrated framework and full community simulations. Plasticity mechanisms, but also plasticity as a strategy (look at the cost and tau). 

Effects on productivity and coexistence. Difference in the correlation ?

Effect of tau on persistence.

\section{Plasticity and diversity}

Now 

\subsection{Method}

\paragraph{Simulations}
To test the effect of plasticity on coexistence and community dynamics, runs from the parameter filtering are used as starting points to limit the simulation time of the stabilisation phase. For each parameter set tested, 6 different sites were tested during the calibration phases, 77 parameter sets were accepted, resulting in 462 communities. Each of those is the starting point of three parallel runs that differ only by the allocation algorithm used: \textit{non plastic}, \textit{fixed-equilibrium} and \textit{plastic-optimisation}. The \textit{fixed-equilibrium} is favoured to \textit{fixed-optimisation} algorithm because previous part of the document focused on this algorithm and because it is simpler to analyse. The \textit{plastic-optimisation} algorithm is still simulated, despite the relatively poor performance results observed in constant conditions and the high convergence, because the introduction of plasticity cost, continuous species specific plasticity ($0 < \tau < 1$), and temporal and spatial heterogeneity should mitigate the negative sides of this allocation mechanism and give information of processes at stake.




\subsection{Results}

Need to run the simulations. Script is almost ready, parameters are filtered from previous step.

\paragraph{General behavior}
\paragraph{Plasticity: a winning strategy ?}
Are plastic species more selected than the other ? Probably a bell-shape curves

\paragraph{Effect on coexistence}

\paragraph{Plasticity selection}

Not done yet. when is plasticity selected: is there a correlation between environment variables (or variations and plasticity selection). trade-off with other traits ?

\section{Strategy}

\subsection{Results}
\paragraph{Changes in strategy}

Select different strat?

Lead to changes in strats? If yes, is it directional, or is the direction depends on species?
Do species change a lot there strategies?

Is it always in the same direction for all species ? (reproduce Kichenin).

\subsection{Discussion}
\paragraph{shift in strategy}


also, Jung \cite{jung_intraspecific_2014} show contrasting response between species and within species - might not be the best 
