

%_________________________________________________________________________________
\chapter{Outlook}

Further reading, thinking and rambling about what's developped in the papers.

%\section{Plasticity and resistance to climatic events}

%\section{•}

\subsection{Better calibration}

Better implementation rcpp or data table struture, plasticity mech, to allow bayesian and pattern-oriented calibration. A lot of species and a lot of parameters. Difficult exercice. But strategies should be limited by functions and trade-off, just need to calibrate shared (process related) parameters and not species specific (strategic) paramters. This is what allows the modelling of diverse community. 


\section{Competition and feedback}
This document focuses on how the plant are doing with the given resources (arrow in fig in margin). However, a key element in competition and resource dynamics (point that separate Tilman appraoches from Chesson) is the impact of plant on resource (fig in margin). Both are fundamental for the understanding on plant interactions, and I argue that understanding the former is necessary to understand the later and have a global view on plant competitive interactions on resources. blablabla competition experiments, resistance to resource shortening (Tilman) and relative homogeneity of resource (homogeneous in influx, content, starting pool, ... ?). Using the term homogeneous allows to use fixed terms and processes, while to me there is a ambiguity around competition that can be seen as: (1) the impact on growth, (2) the winner out of a competitive scenario (with resource shortening). In this later case, the approach of part 4 (?) has limited interpretation since they are not competing. We can intuiitively imagine (from our understanding of model's functioning) that  there is a hierarchical effect on growth, but that is probably reversed in case of (1) shared resource pool (big plant may have access to bigger resource pool in open environment), (2) sufficiently quick resource shortening to lead to death events.\\

in margin: figure resource and interaction.\\
figure competition decomposition of fitness (growth and survival), and growth related to resource pool (try to have graph approach).

competition change vegetation response to climate change \parencite{van_loon_how_2014}

transitivity and competition \cite{levine_beyond_2017} Could it emerge from the current implementation of \model ? Is it stable with plasticity ?


\section{Extend to climate change effects}

How plasticity actually affect the effects of climate change: mitigate or amplify, risk of critical transition.

drought resistance experiments to be done.

Higher diversity: higher risk of invasion?

Take advantage of simulated scenarios of climate change.

\section{Going forward: epigenetic and heritability}

fundamental knwowledge 

effect of heritability and genetic effects (evolutionary perspective) Bring the two perspective together. \cite{scheiner_genetics_1989}

Bayesian model of dev. \cite{stamps_bayesian_2016} Talked about the difficulty to match reaction norms with systemic plasticity: evolutionay bayesian approach to species specific reaction norms.

Epigenetic variation creates potential for evolution of plant phenotypic plasticity \cite{zhang_epigenetic_2013}
See also \cite{dewitt_expanding_2016} for higher moment of reaction norms controlled by genes