

%_________________________________________________________________________________
\chapter{Synthesis}

Point out the novelty, acheived work 

begin to fill the gaps of community dynamics: \parencite{berger_competition_2008}: effect on local environment, adaptive behaviour and below-ground.

Pllus: lack of diversity.

%\section{Phenotypic plasticity: the individual response alter}
\section{Modelling diverse community}

It grows.

It's diverse.

It's stable.

It's driver dependant? (at least at individual level.


\section{Effect of plasticity of mountain grasslands properties}

Did a bit what I accused other model to do: have a discrete conception of plasticity. However, the results at the community scale are encouraging and demonstrate the interest of such approach.
%
%\paragraph{Interpretation}
%In modelling studies it is necessary to realised that we are not looking at the real system, but rather comparing models. Therefore, when we look at the results of model including or not plasticity, if we are looking at the direct effect on the model, outcome change, while there do not change in reality. So the interpretations of the effects on the model behaviour must be translated, not as effects on real the community, but as effects on how we understand the processes shape this community. An increase in individual growth rate due to plasticity is consequently translated in "we overestimate growth parameters in non palstic growth models". <<-- Not sure it is that interesting. Maybe talk about the weight between processes, rather than parameters that are calibrated and are less linked to the understanding of the system.

\subsection{Identity}
\subsection{Productivity}
\subsection{Diversity}

\section{On plasticity modelling}
One strong assumption this modelling relied on was the existence of a strong link between fitness and environnemental condition. This is has been proven to be partially true as \model was able to express improvements in fitness thanks to plasticity. However, in some situations, the plasticity leads to reduction in fitness, or eventually to complete phenotypic dead-end. The temporal dimension of plant growth, and the difficulty to capture that makes this assumption hard to maintain in such complex systems with strong dyunamics. Moreover, such assumption do not necesserally take into account competitive behaviours better capture by game theory and other modelling approaches \cite{farrior_resource_2011, dybzinski_evolutionarily_2011}.


\subsection{How to make it work better, with what consequences} 
\paragraph{On memory}

Despite talking about molecular basis of the plasticity, did not really make use of this knowledge. Should better use biological idea (even if a bit more complex). Especially for memory and recovery: idea of stress, memory and loss of memory (recover) to avoid maladaptive responses \parencite{crisp_reconsidering_2016} ! ! ! 
seee also cues reliability \parencite{simons_playing_2014} and \parencite{scheiner_genetics_1989, scheiner_genetics_2002, scheiner_genetics_2012, scheiner_genetics_2013}

\paragraph{On plastic traits}

Non composite traits (here SLA plastic only because of density, but does not consider thickness plasticity).

\paragraph{On drivers}

Fitness proxy ? yeah \cite{ryser_consequences_2000} resource use optim by pl, but \cite{franklin_modeling_2012} too variable ... stess response oriented. 


Nitrogen instead of water, leaves do not respond to water changes (unless low nitrogen becaue water limitation is more nitrogen limitation really \parencite{farrior_competitive_2014}), does not work well with integrative vision to ignore nitrogen.


\subsection{Genericity and extensions}

About extensions, but the process, despite failing completly capture real growth dynamics (but we already disccused why), built the fundations for resistance-risk plasticity with accumulation of risk cues, extend risk avoidance - resistance trade-off (similar than droudgt see \cite{kooyers_evolution_2015}?) for herbivory.

\section{The limit of the species.}

Refer to the litterature review part.

In this work, but also because of the improvement of molecular biology, and the deeper and deeper dive ecology is doing within individual, the limits of species are fuzzy (started with trait and the introduction of continuity). At some point, there will be a need for a way to go back from the a space of numerous continuous dimensions to the species. Also, understanding species as evolving 3D objects, where the different aspects of intra-specific variations play different shaping roles.