

%_________________________________________________________________________________
\chapter{Synthesis}

Point out the novelty, acheived work 

begin to fill the gaps of community dynamics: \parencite{berger_competition_2008}: effect on local environment, adaptive behaviour and below-ground.

Pllus: lack of diversity.

%\section{Phenotypic plasticity: the individual response alter}
\section{Modelling diverse community}

\section{Effect of plasticity of mountain grasslands properties}

\subsection{Identity}
\subsection{Productivity}
\subsection{Diversity}

\section{On plasticity modelling}
One strong assumption this modelling relied on was the existence of a strong link between fitness and environnemental condition. This is has been proven to be partially true as \model was able to express improvements in fitness thanks to plasticity. However, in some situations, the plasticity leads to reduction in fitness, or eventually to complete phenotypic dead-end. The temporal dimension of plant growth, and the difficulty to capture that makes this assumption hard to maintain in such complex systems with strong dyunamics. Moreover, such assumption do not necesserally take into account competitive behaviours better capture by game theory and other modelling approaches \cite{farrior_resource_2011, dybzinski_evolutionarily_2011}.

\section{The limit of the species.}

Refer to the litterature review part.

In this work, but also because of the improvement of molecular biology, and the deeper and deeper dive ecology is doing within individual, the limits of species are fuzzy (started with trait and the introduction of continuity). At some point, there will be a need for a way to go back from the a space of numerous continuous dimensions to the species. Also, understanding species as evolving 3D objects, where the different aspects of intra-specific variations play different shaping roles.