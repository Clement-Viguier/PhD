

%_________________________________________________________________________________
\chapter{Synthesis}

Point out the novelty, acheived work 

%\section{Phenotypic plasticity: the individual response alter}

\section{Competition and feedback}
This document focuses on how the plant are doing with the given resources (arrow in fig in margin). However, a key element in competition and resource dynamics (point that separate Tilman appraoches from Chesson) is the impact of plant on resource (fig in margin). Both are fundamental for the understanding on plant interactions, and I argue that understanding the former is necessary to understand the later and have a global view on plant competitive interactions on resources. blablabla competition experiments, resistance to resource shortening (Tilman) and relative homogeneity of resource (homogeneous in influx, content, starting pool, ... ?). Using the term homogeneous allows to use fixed terms and processes, while to me there is a ambiguity around competition that can be seen as: (1) the impact on growth, (2) the winner out of a competitive scenario (with resource shortening). In this later case, the approach of part 4 (?) has limited interpretation since they are not competing. We can intuiitively imagine (from our understanding of model's functioning) that  there is a hierarchical effect on growth, but that is probably reversed in case of (1) shared resource pool (big plant may have access to bigger resource pool in open environment), (2) sufficiently quick resource shortening to lead to death events.\\

in margin: figure resource and interaction.\\
figure competition decomposition of fitness (growth and survival), and growth related to resource pool (try to have graph approach).

\section{On plasticity modelling}
One strong assumption this modelling relied on was the existence of a strong link between fitness and environnemental condition. This is has been proven to be partially true as \model was able to express improvements in fitness thanks to plasticity. However, in some situations, the plasticity leads to reduction in fitness, or eventually to complete phenotypic dead-end. The temporal dimension of plant growth, and the difficulty to capture that makes this assumption hard to maintain in such complex systems with strong dyunamics. Moreover, such assumption do not necesserally take into account competitive behaviours better capture by game theory and other modelling approaches \cite{farrior_resource_2011, dybzinski_evolutionarily_2011}.

\section{The limit of the species.}

Refer to the litterature review part.

In this work, but also because of the improvement of molecular biology, and the deeper and deeper dive ecology is doing within individual, the limits of species are fuzzy (started with trait and the introduction of continuity). At some point, there will be a need for a way to go back from the a space of numerous continuous dimensions to the species. Also, understanding species as evolving 3D objects, where the different aspects of intra-specific variations play different shaping roles.