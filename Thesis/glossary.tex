



\newglossaryentry{ecosystem services} 
{
name = ecosystem services,
description = {Ensemble of services provided by an ecosystem that benefit humans. Carbon storage, water purification, recreational value or wood production are all examples of ecosystem services.} 
}


\newglossaryentry{ecosystem} 
{
name = ecosystem,
description = {Both living and non living components of a systems binded together by interactions.} 
}



%##########################################################################
\newglossaryentry{plasticity}{name={Plasticity},description={\glspar}}

\newglossaryentry{active plasticity}
{
    name=active plasticity,
    description={Change in phenotype controlled by internal regulation processes. Opposed to passive response. \textit{i.e.} change in SLA when light is limiting is an active plastic response.},
    parent = plasticity
    }


\newglossaryentry{adaptive plasticity}
{
    name=adaptive plasticity,
    description={Active plastic response that lead to an increase in fitness. A plastic response may be adaptive despite an apparent reduction of fitness, indeed the adaptive response may mitigate a larger reduction in absence of plastic response.},
    parent = plasticity
    }

%
%\newglossaryentry{plasticity}
%{
%    name=plasticity,
%    description={Capacity of one genotype to generate multiple phenotype.},
%    parent = plasticity
%}


\newglossaryentry{static gain}
{
    name=static gain,
    description={The static gain refers to a gain in biomass or fitness due to a change in phenotype, from a default phenotype to a new \textbf{static} phenotype that has greater fitness. Any non plastic plant with the said phenotype would have the same fitness (or higher because of plasticity cost and delay). It is opposed to the dynamic gain.}
    }
    
\newglossaryentry{dynamic gain}
{
    name=dynamic gain,
    description={The dynamic gain refers to a gain in biomass or fitness due to changes in phenotype during time. These changes follow changes in the optimum phenotype due to temporal variations of the conditions. It is opposed to the static gain.}
    }


\newglossaryentry{allocation rule}
{
    name=allocation rule,
    description={The allocation rule is the set of rules that determine the target phenotype of a plant considering its actual phenotype, the biomass available and the projection of external conditions. It can be decomposed in two main parts: the plastic dimensions, and the fitness proxy function (or gain function). Allocation rule is also designated as allocation algorithm, plasticity rule or plasticity algorithm.}
}