\documentclass[review]{elsarticle}

\usepackage{lineno,hyperref}
\modulolinenumbers[5]

\journal{Journal of \LaTeX\ Templates}

%%%%%%%%%%%%%%%%%%%%%%%
%% Elsevier bibliography styles
%%%%%%%%%%%%%%%%%%%%%%%
%% To change the style, put a % in front of the second line of the current style and
%% remove the % from the second line of the style you would like to use.
%%%%%%%%%%%%%%%%%%%%%%%

%% Numbered
%\bibliographystyle{model1-num-names}

%% Numbered without titles
%\bibliographystyle{model1a-num-names}

%% Harvard
%\bibliographystyle{model2-names.bst}\biboptions{authoryear}

%% Vancouver numbered
%\usepackage{numcompress}\bibliographystyle{model3-num-names}

%% Vancouver name/year
%\usepackage{numcompress}\bibliographystyle{model4-names}\biboptions{authoryear}

%% APA style
%\bibliographystyle{model5-names}\biboptions{authoryear}

%% AMA style
%\usepackage{numcompress}\bibliographystyle{model6-num-names}

%% `Elsevier LaTeX' style
\bibliographystyle{elsarticle-num}
%%%%%%%%%%%%%%%%%%%%%%%

%% Custom commands:
\newcommand{\model}{\textbf{\textit{MountGrass}}~}




%%%%%%%%%%%%%%%%%%%%%%%

\begin{document}

\begin{frontmatter}

\title{General framework for coexistence including phenotypic plasticity: the model \model}
\tnotetext[mytitlenote]{Fully documented templates are available in the elsarticle package on \href{http://www.ctan.org/tex-archive/macros/latex/contrib/elsarticle}{CTAN}.}

%% Group authors per affiliation:
\author{Elsevier\fnref{myfootnote}}
\address{Radarweg 29, Amsterdam}
\fntext[myfootnote]{Since 1880.}

%% or include affiliations in footnotes:
\author[mymainaddress,mysecondaryaddress]{Elsevier Inc}
\ead[url]{www.elsevier.com}

\author[mysecondaryaddress]{Global Customer Service\corref{mycorrespondingauthor}}
\cortext[mycorrespondingauthor]{Corresponding author}
\ead{support@elsevier.com}

\address[mymainaddress]{1600 John F Kennedy Boulevard, Philadelphia}
\address[mysecondaryaddress]{360 Park Avenue South, New York}

\begin{abstract}
\end{abstract}

\begin{keyword}
\texttt{elsarticle.cls}\sep \LaTeX\sep Elsevier \sep template
\MSC[2010] 00-01\sep  99-00
\end{keyword}

\end{frontmatter}

\linenumbers

\section{Introduction}
There is a need for community models capable of reproducing diverse plant communities. To investigate the effect of climate change it has to incorporate mechanism of response and individual level.\\
Such mechanism is called phenotypic plasticity\\

\section{Methods}
\subsection{Model overview}
\subsection{Calibration data}
\subsection{Simulation setup}

\section{Results}
\subsection{Growth of diverse species}
calibration\\
sensitivity analysis\\
1d gradient: distribution of as and memory of surviving species.\\
niche\\

\subsection{Plasticity in this framework}
compare algorithm.\\
effect of tau on growth (same parameters but with no plasticity cost)\\
plastic calibration\\


\section{Discussion}

\section{Conclusion}
using fundamental "deep" traits and memory: able to reproduce a diversity of resource use strategies. Possibility to \\
%These strategies have an impact on niche and interactions\\
This framework is compatible with phenotypic plasticity\\
plasticity change the niche shape (and probably interactions) and may have an impact on community dynamics.


\section*{References}
\cite{Dirac1953888}
\bibliography{201703_bib_paper1}

\part*{Appendices}

%\section{Theory}

\section{\model description}

\section{State variables, traits and parameters}
\subsection{State variables}
\subsection{Species specific traits}
\subsection{Parameters}

\section{Simulations}


\end{document}