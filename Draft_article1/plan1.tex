\documentclass[review]{elsarticle}

\usepackage{lineno,hyperref}
\modulolinenumbers[5]

\journal{Journal of \LaTeX\ Templates}

%%%%%%%%%%%%%%%%%%%%%%%
%% Elsevier bibliography styles
%%%%%%%%%%%%%%%%%%%%%%%
%% To change the style, put a % in front of the second line of the current style and
%% remove the % from the second line of the style you would like to use.
%%%%%%%%%%%%%%%%%%%%%%%

%% Numbered
%\bibliographystyle{model1-num-names}

%% Numbered without titles
%\bibliographystyle{model1a-num-names}

%% Harvard
%\bibliographystyle{model2-names.bst}\biboptions{authoryear}

%% Vancouver numbered
%\usepackage{numcompress}\bibliographystyle{model3-num-names}

%% Vancouver name/year
%\usepackage{numcompress}\bibliographystyle{model4-names}\biboptions{authoryear}

%% APA style
%\bibliographystyle{model5-names}\biboptions{authoryear}

%% AMA style
%\usepackage{numcompress}\bibliographystyle{model6-num-names}

%% `Elsevier LaTeX' style
\bibliographystyle{elsarticle-num}
%%%%%%%%%%%%%%%%%%%%%%%

%% Custom commands:
\newcommand{\model}{\textbf{\textit{MountGrass}}~}




%%%%%%%%%%%%%%%%%%%%%%%

\begin{document}

\begin{frontmatter}

\title{General framework for coexistence including phenotypic plasticity: the model \model}
\title{The individual basis of plant coexistence in mountain grassland and the effect of phenotypic plasticity: investigation with the model \model}
\tnotetext[mytitlenote]{Fully documented templates are available in the elsarticle package on \href{http://www.ctan.org/tex-archive/macros/latex/contrib/elsarticle}{CTAN}.}

%% Group authors per affiliation:
\author{Elsevier\fnref{myfootnote}}
\address{Radarweg 29, Amsterdam}
\fntext[myfootnote]{Since 1880.}

%% or include affiliations in footnotes:
\author[clement.viguier@irstea.fr,mysecondaryaddress]{Clément Viguier}
\ead[url]{www.elsevier.com}

\author[mysecondaryaddress]{Global Customer Service\corref{mycorrespondingauthor}}
\cortext[mycorrespondingauthor]{Corresponding author}
\ead{support@elsevier.com}

\address[mymainaddress]{1600 John F Kennedy Boulevard, Philadelphia}
\address[mysecondaryaddress]{360 Park Avenue South, New York}

\begin{abstract}
\end{abstract}

\begin{keyword}
\texttt{elsarticle.cls}\sep \LaTeX\sep Elsevier \sep template
\MSC[2010] 00-01\sep  99-00
\end{keyword}

\end{frontmatter}

\linenumbers

\section{Introduction}

Global change has been subject of a large, and still growing, number of studies. Yet, because the complexitiy of ecological system coupled with the uncertainty around the future of climate and management, a lot of work is remaining to predict the state of natural and semi-natural systems in the future. Vegetation communities are of particular interest as they provide both economic value and ecosystem services. If a large part of plant community ecology is foccused on forests, the presumed vulnerability to global change of mountain grasslands has led scientists to study them. If their actual vulnerability is still discussed (phd of sandra, ecoveg 2015), mountain grasslands will certainly be exposed to increasing temperature and droughts, but also to changes in management practices with a reduction of grazing (ask greg, see ref in ceres baros first paper).\\
To better understand and predict the effect of global change in these ecosystems of mountain grasslands, empiricall studies and experiments have been set. (see Jena, leca and Levine). These studies and others (Violle, albert, jung) highlight the importance of intra-specific variations in community ecology. Intra-specific variations represent around 20 and 30 percent of total variation in grasslands (see albert) and could greatly alter the species interactions and community response to abiotic factors. Considering intra-specific variations is important to better understand community dynamics (violle) and in models because they favour coexistence (Clark, Jung, Courbaud) and modify community responses (Jung). Such effects are succeptible to greatly influence the dynamic of communities facing global change by mitigating species level response, soften plant niches frontier, altering species competition. Another argument is the fact that intra-specific changes may alter directly the community response to a stress (Jung).\\

 Moreover the role of long term evolutionary and ecological processes cannot be easilly assessed in such designs. Moreover, considering the multiplicity of climatic and management scenarios scales up the work to a limiting point.\\



To overcome these limitations and difficulties, modelling approaches has been developped (fate-h, samsara, taubert, lohier ...). They are either used for the retrospective studies long term dynamics from time series data, the prediction of community dynamics along different scenarios, or to interogate the underlying mecanisms of community dynamics (gemini). These models, to be able to account for changing condittions are all based on strong plant functioning processes at the scale of interest and are supported by field data through parametrisation.\\


Most of these models feature a fixed plant functioning where the dynamic of the community is mostly driven by, 1) the abiotic conditions, 2) the relative competitiveness of species/group specific parameters or direct competion coefficients. If 1) is essential in the context of climate change, the point 2) as main mecanism of plant interaction can be discussed as it relies on differences between species specific physiological parameters. Physiological or competition parameters are generaly estimated by direct measures or derived from dta through calibration. Both methods give estimators in what we have good confidence, however they do not allow them to vary within the group (plant functional type, species or population) they haved been difined for. The estimator produce good results and generally follow fundamental or ecological trade-offs. Yet, they do not allow for variations within the group for these parameters, as they are not strictly constraint by said trade-offs and could lead to darwinian demons, or would require calibration of this variation space. More efforts must be done in the representation of the link  between chemical, anatomical and morpholocical traits and physiological traits that drive plant growth and plant interactions. Defining such link would authorise variations within a group while maintaining strong trade-offs between physiological traits and allowing variation and search in the strategy space... (not clear).\\

We stressed the importance of considering intra-specific variations, ann highlight the necessity for a link between chemical and anatomical traits to functioning traits.  
 Not clear what is genetic variation and selection/evolutionary processes or phenotypic plasticity. Phenotypic plasticity in models: theoretical: 2 species interactions, not at community model. (Heritability ?)




There is a need for community models capable of reproducing diverse plant communities. To investigate the effect of climate change it has to incorporate mechanism of response and individual level.\\
Such mechanism is called phenotypic plasticity\\

\textbf{
How is it really different from source-sink approaches ? Or functional equilibrium ?}
-> allocation based, trait variations, plasticity is a strategy. This last poinnt is important. It's related to the discussion around van kleunen ad Dewitt (not all species are plastic, sultan says most are) <- check that.

\section{Methods}
\subsection{Model overview and concepts}
\paragraph{Overview of \textit{MountGrass}}
\paragraph{Plasticity in \textit{MountGrass}: concepts and implementation}
\textbf{Allocation} Why allocation and not just traits ? Allocation model provide structural constraint for plant strategies. Study ecology is studying the relative performance of individuals (and their impact on environmental conditions) in relationship with their strategies. Considering the amount of traits and strategies plants can develop, it is crucial to reduce the dimensionality of the strategic space (space define by all independent strategic axis plant can be found on). The most effective way to use laws of physics, chemistry and biology to eliminate impossible strategies (or combination of traits). Allocation based model take advantage of the "law of mass balance" ... to limit the number of possible allocation pattern, or strategies. This approach has the advantage of creating limited \textit{continuous} strategy space that can be explored and reveal ecological relationships/constraint. The search for such relationships or trade-offs is a big challenge in empirical ecology (see \cite{reich_evolution_2003, wright_worldwide_2004, diaz_plant_2004, diaz_global_2016} for plants) and ecological modelling (see \cite{reineking_environmental_2006, falster_plant:_2016}) as they reduce the complexity and help understand the main mechanisms that shape communities.\\

\textbf{Plasticity}: expected environment -> phenotype, here phneotype is equivalent to biomass partitioning, that means expected environement -> allocation coefficients. Then memory -> expectations -> allocation. Because low dimensions, and we want diversity, and the link between memory and allocation might not be a function (one memroy give exactly one optimum allocation), in the model this relationship is not verified. Species specfic traits are used to allow for different strategies to be associated to a same memory (different plants won't have the same strat, despite sharing the projection)

\textbf{Allocation algorithms}


\subsection{Calibration}
\paragraph{Pot data}
Pot data consists in total biomass and root shoot ration (RSR) data of ... species grown in pots by Peterson and al. (peterson). This old dataset has the advantages of being grass species grown in a described steady environment with two conditions of watering with measures of essential components of growth: biomass and RSR. The inputs used to simulated these experriment are detailed in appendix.

\paragraph{Individual calibration process}
Bayessian calibration could not be used for the model considering the number of parameters and the simulation time. A filtering process has been implemented in R. Parameters are sampled following the LHS method (from \texttt{lhs} package)	within parameter ranges (desccribed in table ...) defined from the litterature, and constraints dicted by desired behaviours from the model. When necessary the sample is log transformed. Because of strong relationship between exchange rate parameters and cost of exchagne area, exchanges rates parameters are expressed on a mass basis for sampling then transform to an area basis for the model. Phtosynthetic activity is defined relatively to the water uptake activity and water use efficiency (WUE) to avoid extreme root shoot ratios.\\

Once generated a first filtering is applied to save simulation time and avoid unrealistic trait values (see table for ranges extracted from LES data in alpine biome) that are not tested against calibartion data.\\
Once the parameters transformed and filtered, simulations matching growth conditions in Peterson experiments.\\
Generated data from finished simulations (i.e. plant lives until the end and do not exceed model's internal size limits) are then compared to experiment data species by species. Parameters of logistic distribution are computed from species means and standards deviantions for RSR and total biomass. The use of this distribution form is justified by the intrinsic form of RSR measure and the need to reject negative values for total biomass. A parameter set is accepted for one species if it within a 95\% range of the calculated distribution for both RSR and total biomass in wet and dry conditions.\\

\paragraph{Field data}
Field data has been collected between years 201 .. and 201 in two distinct datasets from Chalmandrier and al.() and Claire Deleglise and al. (). 
\subsection{Simulation setup}

\section{Results}
\subsection{Growth of diverse species}
calibration. Calibration filtering results in the selection of n parameter sets over m preselected parameters sets. Accepted sets are distributed among the 11 species of the dataset like presented in the table. Species A, B and C are the most numerous.\\
sensitivity analysis. The models about seems to be sensitive to the following parameters: . Total biomass is particularly sensitive to exchange rate parameters, but also tissue construction cost.\\

Plasticity and acceptance rate\\
Change of relationship between parameters and acceptance rate -> none\\
accept = f(tau)

1d gradient: distribution of as and memory of surviving species.\\
niche\\

\subsection{Plasticity in this framework}
compare algorithm.\\
effect of tau on growth (same parameters but with no plasticity cost)\\
plastic calibration\\


\section{Discussion}

\section{Conclusion}
using fundamental "deep" traits and memory: able to reproduce a diversity of resource use strategies. Possibility to \\
%These strategies have an impact on niche and interactions\\
This framework is compatible with phenotypic plasticity\\
plasticity change the niche shape (and probably interactions) and may have an impact on community dynamics.


\section*{References}
\bibliography{../Bibliography/bib_zotero20171106}

\part*{Appendices}

%\section{Theory}

\section{\model description}

\section{State variables, traits and parameters}
\subsection{State variables}
\subsection{Species specific traits}
\subsection{Parameters}

\section{Simulations}


\end{document}