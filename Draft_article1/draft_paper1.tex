\documentclass[review]{elsarticle}

\usepackage{lineno,hyperref}
\modulolinenumbers[5]

\journal{Journal of \LaTeX\ Templates}

%%%%%%%%%%%%%%%%%%%%%%%
%% Elsevier bibliography styles
%%%%%%%%%%%%%%%%%%%%%%%
%% To change the style, put a % in front of the second line of the current style and
%% remove the % from the second line of the style you would like to use.
%%%%%%%%%%%%%%%%%%%%%%%

%% Numbered
%\bibliographystyle{model1-num-names}

%% Numbered without titles
%\bibliographystyle{model1a-num-names}

%% Harvard
%\bibliographystyle{model2-names.bst}\biboptions{authoryear}

%% Vancouver numbered
%\usepackage{numcompress}\bibliographystyle{model3-num-names}

%% Vancouver name/year
%\usepackage{numcompress}\bibliographystyle{model4-names}\biboptions{authoryear}

%% APA style
%\bibliographystyle{model5-names}\biboptions{authoryear}

%% AMA style
%\usepackage{numcompress}\bibliographystyle{model6-num-names}

%% `Elsevier LaTeX' style
\bibliographystyle{elsarticle-num}
%%%%%%%%%%%%%%%%%%%%%%%

%% Custom commands:
\newcommand{\model}{\textbf{\textit{MountGrass}}~}




%%%%%%%%%%%%%%%%%%%%%%%

\begin{document}

\begin{frontmatter}

\title{General framework for coexistence including phenotypic plasticity: the model \model}
\tnotetext[mytitlenote]{Fully documented templates are available in the elsarticle package on \href{http://www.ctan.org/tex-archive/macros/latex/contrib/elsarticle}{CTAN}.}

%% Group authors per affiliation:
\author{Elsevier\fnref{myfootnote}}
\address{Radarweg 29, Amsterdam}
\fntext[myfootnote]{Since 1880.}

%% or include affiliations in footnotes:
\author[mymainaddress,mysecondaryaddress]{Elsevier Inc}
\ead[url]{www.elsevier.com}

\author[mysecondaryaddress]{Global Customer Service\corref{mycorrespondingauthor}}
\cortext[mycorrespondingauthor]{Corresponding author}
\ead{support@elsevier.com}

\address[mymainaddress]{1600 John F Kennedy Boulevard, Philadelphia}
\address[mysecondaryaddress]{360 Park Avenue South, New York}

\begin{abstract}
\end{abstract}

\begin{keyword}
\texttt{elsarticle.cls}\sep \LaTeX\sep Elsevier \sep template
\MSC[2010] 00-01\sep  99-00
\end{keyword}

\end{frontmatter}

\linenumbers

\section{Introduction}
%\indent One of the main challenges of community ecology has been to explain the observed diversity despite few theoretical supports to high coexistence, as know as the "plankton paradox". During the last 25 years, our theoretical understanding of this apparent paradox is nourished by empirical and modelling work. This progress partly arose from a shift between approaches focusing on species to the use of functional traits. Functional traits allows the investigation of broader patterns and mechanisms shaping natural communities. Traits emerged s a great tool to link theoretical concepts to empirical studies. The rise of large trait datasets and online trait data repositories favour the investigation of broad general patterns of response to climate, trade-offs, community composition. This shift also introduced the concept on functional diversity that focused more on the functional role of the species in the ecosystem than on their identity.\\
%% talk also about use of traits to niche filtering and climatic niche. Response to climate. Interactions.
%%MRH etc, see nequist for inspiration. + lavorel and diaz. In addition to a better understanding of ecological interactions, plant functional traits are the basis of the assessment of ecosystem services.\\
%\indent Despite the advances made in community ecology by the use of functional trait, certain mechanisms are still misunderstood and the mysteries of coexistence are not yet resolved. This limitation of the approach may lie in the use of the species average traits of species. According to Violle and al. (check the ref) intra-specific trait variability (ITV)... Intra-specific trait variability arises from three main mechanisms (see box 1 for details). Incorporating this intra-specific variability in community models in necessary to enhance our understanding of community assemblage rules and community responses to biotic and abiotic drivers. ITV has been shown to have contrasted effects on conclusions we can make from community models. Jung et al. demonstrated the positive effect of intra-specific variability on coexistence by mitigating the abiotic filtering effect and improving the niche partitioning effect (check that). In the other hand the ITV are supposed to reduce predicted coexistence through Jensen inequality effect (on both XXX function and competitive effects, see for details), despite a light positive impact of released intra-specific competition of low density species. The relative impact of these stabilizing (better niche partitioning, lower intra-specific competition at low density, etc...) and destabilizing () effects of intra-specific variations cannot be resolved by the use of theoretical models only. Empirical works show interesting correlations between shoot plasticity and diversity, but this relationship is only explained by an increase in ramet density and not a increase in diversity per ramet (sign of increased niche partitioning). Disentangle effects, complexity, tractable, generalization... Differences of response 
%Between the complexity of natural systems and the oversimplification of theoretical models, there is a gap to fill if we want to better understand the role of phenotypic plasticity as a regulator of plant responses to climate drivers and biological interactions.\\
%
%\indent To attempt to fill the gap between empirical approaches and theoretical we try to address the following questions: 
%%can the diversity of plastic strategies be reproduced by a generic model of phenotypic plasticity ?
%Can a simple shared plasticity functioning explain the diversity of plant plastic allocation strategies ?
%How the phenotypic plasticity alter species potential niches and species interactions ?
%What is the impact of plasticity costs on the plant strategies and interactions ?\\
%
%\indent In order to tackle these questions we developed a generic process-based model of mountain grasslands. This model 
%rely on a generic framework of plant functioning based on multidimensional strategies and allocation trade-offs, extended with phenotypic plasticity. The model is mostly in this paper at the individual scale to settle a good understanding of phenotypic plasticity at the lowest scale. Work at the higher scale will be able to extend from this basis thanks to community scale simulations. Hopefully this ensemble of work will nourish the discussion about the role of intra-specific variation and phenotypic plasticity on community dynamics, response to global changes and evolution.

%\section{Theory}
%\subsection{Multidimensional strategy space}
%
%\subsection{Allocation trade-offs}
%\subsection{Phenotypic plasticity: between parental memory and confidence in individual experience}

\paragraph{plan}


Challenges of ecology: understand coexistence mechanisms to better predict response to drivers (management view). Coexistence is a major question: it's based on the use of resources, the interactions (indirect and direct) between individuals and the adaptation to environmental conditions... <- a bit vague. Plus many other mechanisms: how to include enough of them to maintain coexistence and eventually study them (alternative to experimental approaches)\\
Objective: Better understanding of coexistence mech and community dynamics by looking at effect of plant on resources, and resource levels on plant.\\
Approach: generalist: Functional traits, shared physiology.\\
Why: services, productivity, (productivity-diversity relationship), better understanding, vulnerability...

\paragraph{Alternative plan}

Challenge: coexistence mech, response to drivers, and functional description.\\
Pb: role of intraspecific variation (gen, or plasticity), potential for very different response, + impact on coexistence mechanisms (but also resistance and resilience.\\
Curent state: see Fabio's biblio. too theoric, some empirical approaches, difficulty to predict community scale effect, and combine effect of management and climate variations. Reaction norms.\\
Introduction of mountgrass: coexistence framework for coexistence, allocation trade-off and phenotypic plasticity framework, interaction based on explicit resource use and dynamics.

Understanding the mechanisms of coexistence leading to high diversity of ecological system has long time be a subject of research. From the apparent paradox of numerous species coexisting in a environment with a limited number of resources, theoretical progress has been made in the understanding of involved mechanisms. Among these mechanisms heterogeneity (spatial and temporal) and perturbations play a big role. Trade-offs are also important aspects of coexistence, preventing the existence of darwinian demons and allowing the existence of diverse equivalent (in fitness) strategies. Understanding these mechanisms is crucial in a time where the natural and semi-natural habitats are subjects to great pressure, changing drivers (global changes) and provide a high number of services of which the importance is more and more revealed. Empirical studies and models are already alvailable to interrogate the effect of drivers such as climate and land-use on vegetation communities (wilfried). However empirical studies are costly both in money and time and can be used for a limited number of scenarios. Models offer a good tool to address these questions in diverse scenarios combining climate trajectories and alternative management plans. These models are often limited to a restricted pool of species in non natural habitat or to braod functional types for semi-natural or natural habitats. ...
Need for taking into account intra-specific variability: both to quantify services, but also to understand and predict community dynamics.


\section{The multi-dimensionality of coexistence}
multiplicity of mechanisms (equilizing, stabilizing) and scales: empirically studies are often interested in one specific scale, but plant functioning is not, a generalizing model should try to let the door open to that. We propose here a framework to gain in genericity, but can be adjusted to particular important mechanisms (resistance or avoidance of predation, frost, drought, submersion, ...).\\
Need to draw axis of strategic differenciation : lead to strategic volume. Axis are drawn from fundamental trade-offs, ecological trade-off emerge from the model.\\
Particular effect of 


\section{Modelling phenotypic plasticity}
\subsection{Why ? The advantages (and limits)}
adjust to weekly changes, avoid deadly conditions (to discuss: strategy here can be subtle, but the tau can help). Complexity of fitness: balance between species strategy and minor adjustment. 

\subsection{Principle}
Use a future projection, alternative phenotype and gain function. The gain function allows to rank the alternative phenotypes to the estimated future conditions. The control is made by the projection because controlled by few traits. Discuss the diversity of strategy in the context of climatic conditions leading the phenotype choice, it's kind of paradoxical. But still control through plasticity cost function and some species specific strategy traits. These traits may adjust the gain function. \\
On the importance of the main function: it makes the equilibrium in the process that compose fitness. Fitness is decomposed in multiple aspects: survival and reproduction, both mainly dependant on growth and allocation to specific mechanisms. Balance mechanisms: frost resistance, predation resistance, structural investment, productivity, competition... If we want to model diversity, let's keep it simple, and limits it to the allocation to the simple trade-of described earlier. Let's not talk about morphology.\\
Hard to balance single organ optimisation and full plant optimisation ? Gain could be integrated (especially if you consider decrease in leaf potential). But if you have a balanced design and stable leaf activity, because gain evaluation is instantaneous (day by day), you should not need to integrate.\\
THe role of estimation: you might want to follow global variations of resource vailability, but also keep the strategy consistent: i.e. more or less conservative. Particularly make sense the context of competitive vs stressed environment.

\section{Methods: example of the model MountGrass}
\subsection{Purpose}
The model \model is designed to investigate the effect of phenotypic plasticity on coexistence mechanisms in response to climatic and management drivers on mountain grasslands individual plants and communities. Its generic structure allows the representation of numerous species with diverse strategies. Its design also lets ones use it both as a toy model for plants in controlled conditions (main use case in this paper) or as a community model (after calibration).\\

\subsection{Structure}
The model is an individual, spatially explicit, process-based model.\\
Plants in \model are represented as two cylinders: one for shoot and one of fix depth (soil depth) for roots. These basic shapes are used in the calculation of resource gathering. Individuals are defined by an ensemble of species specific traits (see table ) and state variables (table). Plant phenotype can be derived from model parameters, species specific traits and size of the different carbon pools. The plant phenotype is completed by an ensemble of variables that describe its internal state and strategy. The carbon pool decomposition relies on the general patterns of the Leaf Economic Spectrum and other large scale studies. The limits of the use of general patterns at the local scale in discussed further in this paper. Statistical modelling approach by Shipley supports the role of one particular trade-off to explain the LES: the proportion of cell tissue (related to cell size and number) to cell-wall tissue (cell-wall thickness). This assumption that the allocation between structural (cell-wall) and active (cell) tissues can explain main trade-offs is at the center on the model carbon pool decomposition described in figure . Other carbon pools are: stem, storage and reproduction. %\\
Plants : multiple strategies,2 cylinders, 4 carbon pools.\\
Environment : soil (water parameters, depth), atmosphere (temp, humidity, light)\\
scales and processes.

\subsection{Representation of processes}

Pseudo-code and schedule.
The vegetation model \model is designed for mountain grasslands, because this type of system works on a season basis, so does the model. It runs on a seasonal schedule, seasons being defined as the periods between the first (check that) snow fall of the first semester and the first snow fall of the second semester. On a biological side seasons start with the germination of new plants and end with the loss of aboveground organic matter. Within the season the processes are computed on a daily basis. The model's schedule is presented in the figure and major steps are detailed in the following paragraphs.\\

Seed bank and germination\\

Resources gathering and competition\\
In \model plants compete for two resources: light and water. Competition for resources being the only form of interactions between plants and closely related to plants strategies, it is the main driver of community structure and dynamics. Light competition is simulated by diffusion law and integration of photosynthetic activity over leaf homogeneous shoot layers within each grid cell and summed up \cite{Taubert} to avoid the computational cost and complexity of ray casting methods. The total shoot potential photosynthetic activity is translated into potential evapo-transpiration or water demand. The water demand is divided over the grid cells of the plant rooting zone to compute the water competition per grid cell. In each cell the total water demand (sum of all plant water demand fractions and potential evaporation) is summed up. If the total demand does not exceed the volume of available water, each plant water demand fraction (and evaporation) is extracted, otherwise the available volume is divided proportionally to the individual demand (see appendix for details). The potential deficit caused by competition in some cells is not compensated in non limited portions of the rooting zone. Because the water demand is calculated from the potential photosynthesis, the water uptake cannot exceed the potential evapo-transpiration.\\
The carbon assimilation equals the realised water uptake multiplied by the water use efficiency (WUE) and temperature reduction factor. The gross primary production (GPP) is reduced by the maintenance respiration proportionally to the amount of active tissues and the growth production if it is still positive. Mass of produce organic matter that can be allocated to plant growth, storage or reproduction is given by the net primary production (NPP) just calculated multiplied by an average leaf carbon content parameter (see table).\\

Resource allocation
The organic matter produced must be allocated to the different carbon pools of the plant. This is crucial as it greatly defines the strategy of the individual and its chances of success in reproduction and/or persistence. Juvenile will invest only in development, whereas mature plants will invest in reproduction and/or storage, and eventually maintain some investment in growth to compensate the tissue turn-over. The shift from juvenile to mature can be controlled by different variable (size, total biomass, degree.days) and constitute the only ontogenic control in \model . The organic matter allocated to development can now be divided between four carbon pools corresponding to the active and structural tissues of shoot and roots. This can follow fixed species specific coefficient or be controlled by simple rules of allocation that join species specific traits and individual experience. This part is further discussed in the plasticity paragraph. \\
Plasticity (and costs).\\
reproduction and dispersion
At the end of the growing season\\
\subsection{Initialisation and inputs}
Because of implementation limitations running sensitivity analysis and full calibration is not possible. The model was used with fixed parameter values unless explicitly said (e.g. for plasticity costs analysis) (see appendix for details). The model was mostly used in the following set-ups:
\begin{itemize}
\item[Pot simulation:]
\item[Double pot simulation:] is it necessary ? very similar to pot simulation. Might not be used.
\item[Greenhouse simulation:]
%\item[Grassland simulation]
\end{itemize}

Rewrite the rest:
In the rest of this paper we will use the terms "pot simulation", "double pot simulation" and "greenhouse simulation" to refer  to these set-ups.\\
If not specified or varying the default precipitation and radiance levels where as follow: 2mm per day and 300 Watt per day (10/12 hours). Species traits are default traits (see appendix for details)\\
Maturity and investment to storage or reproductive tissues is ignored to use the total growth as proxy for fitness.


\section{Results}
\subsection{Know your classics}
Reproduction of classical patterns: intermediate perturbation, heterogeneity, storage effect.
-> that should lead to coexistence.\\
Selection processes: selection of more conservative strategies with stressed conditions, selection of  shorter plants when cuts etc...
\subsection{From memory to phenotype}
Show how memory can lead to different phenotype. Why is it important ? Make the latent/low level traits relevant in the approach. Do the parallel with other traits that could be involved: like resistance to frost, herbivory, UV, or WUE regulations etc... Also makes the plasticity relevant if you also show that these phenotypes have better fitness in the corresponding conditions (keep in mind the effect of competition and that the match between memory and realised resource availability means greatest fitness).

\subsection{Plasticity: }
gain in productivity\\
co selection with fast strategies ?\\
selection of plastic species when variable ?


\section*{References}
\cite{Dirac1953888}
\bibliography{201703_bib_paper1}

\part*{Appendices}

%\section{Theory}

\section{\model description}

\section{State variables, traits and parameters}
\subsection{State variables}
\subsection{Species specific traits}
\subsection{Parameters}

\section{Simulations}


\end{document}