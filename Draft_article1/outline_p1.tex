\documentclass[review]{elsarticle}

\usepackage{lineno,hyperref}
\modulolinenumbers[5]

\journal{Journal of \LaTeX\ Templates}

%%%%%%%%%%%%%%%%%%%%%%%
%% Elsevier bibliography styles
%%%%%%%%%%%%%%%%%%%%%%%
%% To change the style, put a % in front of the second line of the current style and
%% remove the % from the second line of the style you would like to use.
%%%%%%%%%%%%%%%%%%%%%%%

%% Numbered
%\bibliographystyle{model1-num-names}

%% Numbered without titles
%\bibliographystyle{model1a-num-names}

%% Harvard
%\bibliographystyle{model2-names.bst}\biboptions{authoryear}

%% Vancouver numbered
%\usepackage{numcompress}\bibliographystyle{model3-num-names}

%% Vancouver name/year
%\usepackage{numcompress}\bibliographystyle{model4-names}\biboptions{authoryear}

%% APA style
%\bibliographystyle{model5-names}\biboptions{authoryear}

%% AMA style
%\usepackage{numcompress}\bibliographystyle{model6-num-names}

%% `Elsevier LaTeX' style
\bibliographystyle{elsarticle-num}
%%%%%%%%%%%%%%%%%%%%%%%

%% Custom commands:
\newcommand{\model}{\textbf{\textit{MountGrass}}~}




%%%%%%%%%%%%%%%%%%%%%%%

\begin{document}

\begin{frontmatter}

\title{Outline paper 1}
\tnotetext[mytitlenote]{Fully documented templates are available in the elsarticle package on \href{http://www.ctan.org/tex-archive/macros/latex/contrib/elsarticle}{CTAN}.}

%% Group authors per affiliation:
\author{Elsevier\fnref{myfootnote}}
\address{Radarweg 29, Amsterdam}
\fntext[myfootnote]{Since 1880.}

%% or include affiliations in footnotes:
\author[mymainaddress,mysecondaryaddress]{Elsevier Inc}
\ead[url]{www.elsevier.com}

\author[mysecondaryaddress]{Global Customer Service\corref{mycorrespondingauthor}}
\cortext[mycorrespondingauthor]{Corresponding author}
\ead{support@elsevier.com}

\address[mymainaddress]{1600 John F Kennedy Boulevard, Philadelphia}
\address[mysecondaryaddress]{360 Park Avenue South, New York}

\begin{abstract}
\end{abstract}

\begin{keyword}
\texttt{elsarticle.cls}\sep \LaTeX\sep Elsevier \sep template
\MSC[2010] 00-01\sep  99-00
\end{keyword}

\end{frontmatter}

\linenumbers


\section{Introduction}

\section{The model}

The active-structural allocation trade-off, base of multi-dimensional strategy space.\\
Spatially explicit light and water competition (with spatial heterogeneity for water).\\

Phenotypic plasticity in allocation: objective functions and plastic dimensions and momery.\\

\section{Results}
\subsection{Calibration}

\subsection{Basic plasticity allocation behaviour} (here or in the Thesis, needed anyway).
Evolution of carbon pools and eventually traits between the different algorithms.

\subsection{Response to a gradient}

Hypothesis: More conservative strategies will have better 
20*20 root strategies,  along 20 water gradient\\
Show the ability of the model to have different optimum depending on the resource availability. Needed for heterogeneity to have a positive impact on functional diversity. The effect is a bit weak, but should be enough.\\
Phenotypic plasticity doesn't change a lot (but already caped for gradient high part).\\

Water availability memory is need for equilibrium, check if there is a good alignment between memory and water availability.\\

there is a little offset toward higher w_ini, probably because invest more in shoot tissues with higher organ efficiency. Lost from not being at the equilibrium should be the same.

\subsection{Study of a parameter space}

strategy space (15^3 strategies) * 20 parameters * 2 conditions (high resources and low resources: 1/4)\\
Does plasticity work ? Could it improve coexistence and limit invasion ?\\
is there an overlapping between the best performer strategies (for a given position in the 2D space, within 0.9-1 relative perf) between the two conditions. Does that change with plasticity ? - plasticity make the plant explore less space and potentially miss niches.\\
Look at overlapping of "equivalent strategies".\\

\subsection{Response to variable environment}
No particular benefit of plasticity but going to a better strategy sub-space. Even between best strategies, plastic plant should have advantage over the non plastic ones.\\

Hyp: the time variability of the resource may lead to different optimum phenotype despite the same average because of evaporation and feedback on resource: the average realised water availability may change.% (look at mean(water_av)). (equivalent to diversity).
This effect should be reduced by plasticity that will promote tissue efficiency over equilibrium.\\

Hyp: plasticity advantage is better perceived in variable environment. Increasing temporal variability of water resource should increase the advantage of plastic plants over best non plastic plant.\\

\end{document}