 \documentclass[english,10pt]{article}
  %\documentclass[english,10pt,twocolumn]{article}
 \usepackage{setspace}
 \onehalfspacing
 \usepackage[utf8]{inputenc}
 \usepackage{lmodern}
 \usepackage[a4paper]{geometry}
 \usepackage{babel}
 \usepackage{wallpaper}
 \usepackage{graphicx} %images
 \usepackage{graphics} %images
 \usepackage{gensymb} %°C
 \usepackage{textcomp}
 \usepackage{multicol}
 \usepackage{lscape}   % paysage
 \usepackage{pdflscape,array,booktabs}%pages du pdf avec tableau en paysage
 \usepackage{geometry}
 \usepackage{multirow} %tableau avec lignes multiples
 \usepackage{fancyhdr}
 %\usepackage[compact]{titlesec} %titres de section compact
 \usepackage{float}
 \usepackage[most]{tcolorbox}
 \definecolor{grey}{gray}{0.95}
 \definecolor{vert}{RGB}{102,194,165}
 \definecolor{red}{RGB}{136,0,21}
 \usepackage{url}
 \usepackage{tocloft}  %table des figures
 \usepackage{hyperref}
% \usepackage{times}
 \usepackage[small,bf,sf,font=sf]{caption}  %gestion des légendes
 \captionsetup[table]{font={stretch=1,sf}}
%Glossaire Need Perl
\usepackage[toc,nonumberlist]{glossaries}
 \makeglossaries
 \usepackage{amsmath,amsfonts,amssymb} %équations, matrices etc...
 \usepackage{cell} %bibliography style
 \usepackage{indentfirst} %indentation des paragraphes en debut de section
% \usepackage[numbered,autolinebreaks,useliterate]{mcode} %insertion de code Matlab framed,
 \geometry{vmargin=2.5cm, hmargin=2cm}
 \hypersetup{colorlinks=true, linkcolor= black, citecolor=black} %gestion lien hypertexte
%\usepackage[rm]{roboto}
 \usepackage[T1]{fontenc}
\usepackage{titlesec}
\usepackage[switch,pagewise]{lineno}

 \newcommand{\auth}{Clément Viguier}
 \newcommand{\model}{\textit{\textbf{MountGrass }}}
 \newcommand{\tlte}{Draft}
 \newcommand{\tltesmall}{Draft \date}
 
 \newcommand{\gl}{\hspace{0.6cm}\textbf } %A revoir

\setlength{\columnsep}{20pt}
 %__________________________________________________________
 \newenvironment{changemargin}[2]{\begin{list}{}{%
 \setlength{\topsep}{0pt}%
 \setlength{\leftmargin}{0pt}%
 \setlength{\rightmargin}{0pt}%
 \setlength{\listparindent}{\parindent}%
 \setlength{\itemindent}{\parindent}%
 \setlength{\parsep}{0pt plus 1pt}%
 \addtolength{\leftmargin}{#1}%
 \addtolength{\rightmargin}{#2}%
 }\item }{\end{list}}
 
 \setlength{\parskip}{0em}
 \captionsetup{belowskip=8pt,aboveskip=0pt}
 
 
 %___________________________________________________________
 
%\renewcommand{\familydefault}{\sfdefault}  
 \renewcommand{\rmdefault}{ptm}
 
 \titleformat{\section}{
 	\sffamily\color{vert}\Large
 }{\thesection}{1em}{}
 
  \titleformat{\subsection}{
 	\sffamily\color{vert}\large
 }{\thesubsection}{1em}{}
 
 % titre du document
 \title{\tlte}
 % auteur du document
 \author{\auth}
 \date{October 2016} %http://en.wikibooks.org/wiki/LaTeX/Title_Creation
 
 
 %Glossaire
 %\newglossaryentry {alios}{name={alios}, description={portion argileuse du sol meuble à solide}}
 %\newacronym{{ademe}{ADEME}{Agence De l'Environnement et de la Maîtrise de l'Énergie}}
 
 
 
 %New paragraph definition avec saut de ligne
\makeatletter
\renewcommand\paragraph{
\@startsection{paragraph}{4}{\z@}
{-1.25ex\@plus -1ex \@minus -.2ex}
{1ex \@plus .2ex}
{\normalfont\normalsize\bfseries}}
\makeatother  
 
 
 %Légende tableaux :
 \newenvironment{captiontable}{
  \begin{small}
 \begin{sffamily}
 \begin{spacing}{1.1}}
 {
 \end{spacing}
 \end{sffamily}
 \end{small}
 }
 
 
 %DEBUT DU DOCUMENT _____________________________________________________________
 \begin{document}
 \pagestyle{fancy}
 \renewcommand\headrulewidth{0.5pt}
 \fancyhead[R]{\color{black}\small\auth - 2016}
 \fancyhead[L]{\color{black}\small\tltesmall}
 \thispagestyle{empty}
% 
% \newpage
% \strut
% \ThisLRCornerWallPaper{1}{../Figures/frontpage.pdf}
% \newpage
 
 \begin{titlepage}
 
 \maketitle
 \centering
 
 \thispagestyle{empty}
 \end{titlepage}
%
% \begin{changemargin}{2cm}{2cm}	%Sommaire
% \vspace*{2cm}
% \thispagestyle{empty}
% \tableofcontents
%\thispagestyle{empty} 
% \end{changemargin}
% 
%\newpage
%
% \begin{changemargin}{2cm}{2cm}	%Liste des figures
% \vspace*{2cm}
% \thispagestyle{empty}
%\listoffigures
%\thispagestyle{empty} 
% \end{changemargin}
%\newpage
%
%
% \begin{changemargin}{2cm}{2cm}	%Liste des tableaux
% \vspace*{2cm}
% \thispagestyle{empty}
%\listoftables
%\thispagestyle{empty} 
% \end{changemargin}
%\newpage

\linenumbers
\modulolinenumbers[5]


\twocolumn{
%\section*{Introduction}
%%The increasing number of available data in functional ecology, from large scale empirical studies, meta-analysis and data-bases, had a great impact on the field. Thanks to the functional traits formalism, information about all sorts of plants have been gathered and analysed in a large scale. This allowed to underlay global patterns and trade-of...
%%better understanding, global patterns, LES, LHS, and others...\\
%%\indent Also variations, intra-specific variations,. Can change the community dynamics, the ecosystem services levels, productivity (compared to no considered).\\
%%
%%\indent We need general framework that integrate that trade-of if we want to investigate community and evolutionary dynamics. = general plant functioning, mechanistic base, no plant specific parameters. There are models (kleydon, scheiter, falster) have done that, but not only trees. Integrate multidimensions and intraspecific variability (here limited to plasticity).\\
%%
%%\indent To understand the role of intra-specific variation and particularly tof phenotypic plasticity on community dynamics and structure, we need a theoretical tool that offer a way to represent diverse strategies in a general framework of plant functioning and integrating a form of plasticity. We propose a new agent based grassland model that incorporate rich multidimensional strategies thanks to mechanistic trade-of.
%\paragraph{State of the art}
%Trade-ofs between traits that allow generalizing frame work.\\
%Used in global and community models of vegetation\\
%Important role of intra-specific variation in traits (25-25\% of total community trait variation) on community structure and dynamic, and potentially on ecosystem services
%
%\paragraph{Problem, limitation, what's next ?}
%grassland community dynamics less explored than forest dynamics in such framework\\
%community model often ignore intra-specific variation\\
%little understanding of phenotypic plasticity effect on community dynamics because few mechanisms for plasticity
%
%\paragraph{next step, our approach}
%use a model in a generalizing framework of plant functioning based on trade-ofs allowing for high strategy diversity\\
%mechanistic approach of the trade-of and plant functioning\\
%integration of plastic allocation between carbon pools to explore trait variation
%
%\paragraph{Model purpose and principle}
%study effect of plasticity on community dynamics\\
%integrate plasticity based on genetic memory and environment perception in a general framework of multidimensional strategies.
%
%\section{Model}
%\paragraph{Overview}
%agent based model of mountain grassland\\
%spatially explicit competition for light and water\\
%multi-dimensional strategic space\\
%new plastic allocation mechanism between major carbon pools based on future estimation\\
%
%\paragraph{General framework}
%model multiple species, with specific strategies\\
%mechanistic basis: physiology, trade-of\\
%allocation trade-of: LES, DVGMs, Shipley, etc...\\
%
%\paragraph{Daily cycle and competition for resources}
%Competition for light\\
%competition for water\\
%production
%
%\paragraph{Allocation of carbon and plasticity}
%conditions of competition\\
%future estimation: memory, maximize return over time...\\
%objective functions\\
%plasticity cost\\
%algorithms
%
%\section{Results}
%\paragraph{One plant growth and development}
%hypothesis: coordination, pot, 
%
%\paragraph{Monoculture: competition and plasticity}
%
%\paragraph{Mixed community: competition}
%
%\section{Discussion}

\section*{Introduction}
Large scale empirical studies (wright, reich), databases (try) and meta-analysis (poorter mouais, other meta-analysis) have greatly improve our understanding of functional traits in plants. Both trade-off between traits (wright and reich) and trait variations with climatic conditions have been studied, and general patterns could be drawn (ref ? Kraft). This understanding of plant functional traits is key in the study of plant communities. Functional trait can be used to better assess the functioning of a community, its diversity, and the services it offers (lavorel and others: diaz ?). More importantly they offer the opportunity to study very different species together and their interactions (Kraft ?, Georges).\\
\indent Classical grassland community models (gemini and others) often consider species as unique entities with particular functioning determined by their parameters values, extracted from empirical studies. Such approaches required a lot of species specific data, and can only be used in the same context as the data. This dependency from specific data (species and habitat) make difficult generalization of the obtained results. Dynamic global vegetation models (DGVMs) have been generalizing the plant behaviour for a while, using shared traits and evolutionary dynamics to predict global ecosystems dynamics. These approaches can now be strengthen by the contribution from functional trait based studies. Indeed, recently trait based models have been developed to investigate community dynamics (Falster). The use of functional traits enables the modelling of multiple species with different strategies to question the role of trade-of in species coexistence in evolutionary models.\\
%\indent General framework for plant functioning through traits and trade-of allows the investigation of plant coexistence mechanisms at large scales (falster). Coexistence mechanisms and community dynamics should also be interrogated at a finer scale where other mechanisms may be involved. This interest for large temporal and spatial scale may be explained by the greater interest of community ecologists for forest than for other type of vegetation communities. There are indeed few models interested in multi-species grasslands. Changing the scope of plant community ecology could benefit to the whole field of community ecology thanks to a better understanding of finer scale interactions and mechanisms.\\
In addition to be useful to generalize plant functioning, functional traits can also be used as a proxy for ecosystem processes and ecosystem services (lavorel). Empirical studies (lavorel, diaz) and some modelling (thuiller ?, ref ?) approaches have used mean traits and community weighted mean trait values to assess the functioning of plant communities. However there is a growing interest in intra-specific variability (diaz) as it represents a relatively high portion of total trait variation (albert) and can have great impact on community response to drivers 'need references). In addition to potentially change the values of community realised trait means (compared to means computed with species mean trait values)(jung) and so affect the ecosystem services provided, intra-specific variability could also change the output of community composition mechanisms (jung, diaz?). Effect on inter- and intra-specific interactions and coexistence mechanisms are also questioned, with various conclusions (theoretical model and empirical study of plasticity). New plant community models should take intra-specific variations into account to better understand its impact on plant interactions, community composition and ecosystem services.\\
%may play an important role in community composition and dynamics... <- Not that good, refine !\\
\indent Incorporating intra-specific variation in vegetation model can be challenging as there are many sources for such variations: genetic variations, epigenetic variations and phenotypic plasticity as the three main sources (see box 1 for details). Integrating the phenotypic plasticity to the model using a framework of generic traits and trade-ofs is necessary to improve our understanding of plant interactions, responses to drivers and impact on community dynamics.\\
\indent We developed \model an agent-based grassland model that integrates phenotypic plasticity in a general multi-strategy framework based on carbon pools allocation and trade-ofs. The general framework of carbon pools and trade-of allows to study different species and diverse communities within a generic set of rules. The phenotypic plasticity relies on the possibility for plants to change the carbon allocation scheme to the different carbon pools based on their species specific strategy and their individual experience of external conditions. The following paper describes the keys elements of the model and how they integrate to model from single plant growth to multiple plant mixture community dynamic.


%Genetic variation within species can often be easily integrated by adding variation within species specific parameters. This source of variation is mainly interesting for studies focused of few focal species and that integrate reproduction and selection dynamics. More conceptual models that use artificial species would take lower benefit from this aspect, unless particular reproduction dynamics are investigated. Epigenetic variations, despite a real potential effect on short successions dynamics are hard to investigate since they are hard to measure in a large scale and make sense only if there is plasticity already included (ref for epigenetic variation in plants).

%Trade-ofs between traits that allow generalizing frame work.\\
%Used in global and community models of vegetation\\
%Important role of intra-specific variation in traits (25-25\% of total community trait variation) on community structure and dynamic, and potentially on ecosystem services
%
%\paragraph{Problem, limitation, what's next ?}
%grassland community dynamics less explored than forest dynamics in such framework\\
%community model often ignore intra-specific variation\\
%little understanding of phenotypic plasticity effect on community dynamics because few mechanisms for plasticity
%
%\paragraph{next step, our approach}
%use a model in a generalizing framework of plant functioning based on trade-ofs allowing for high strategy diversity\\
%mechanistic approach of the trade-of and plant functioning\\
%integration of plastic allocation between carbon pools to explore trait variation
%
%\paragraph{Model purpose and principle}
%study effect of plasticity on community dynamics\\
%integrate plasticity based on genetic memory and environment perception in a general framework of multidimensional strategies.


%\addcontentsline{toc}{subsection}{Box 1: plant phenotypic plasticity}
\begin{tcolorbox}
[breakable, colback=white, colframe=vert, title=Box 1: plant phenotypic plasticity]
\vspace{0.2cm}
\begin{spacing}{1}
"What's in da box !?" - Brad Pitt\\
\paragraph{Sources of variations}
between individuals (genetic, epigenetic, plasticity), and in time (performance, ontogeny, plasticity)
\paragraph{What is phenotypic plasticity ?}
Phenotypic plasticity is often defined as the capacity of individual from the same genotypes to express different phenotypes. trait variation from same genotype (includes epigenetic here, we don't). Capacity for a plant to actively change its organization scheme to match its own individual perception of external conditions.\\
\paragraph{Phenotypic plasticity in grasses}
which traits, what <e can model ?
\paragraph{Phenotypic plasticity in plant models}
 relative plasticity (depends on the focus), not a plant physiological model -> that means some real plastic traits may be fixed, but change the focus, these fixed traits are lower level, the traits that determine the functional trait values -- coherence to the plasticity, not only variation, or hadhoc function with species specific parameters. \\
plant size at maturity vs time for flowering\\
Perception of conditions\\
Plasticity cost:\\


\end{spacing}
\end{tcolorbox}

\section{Model}
\paragraph{Overview}
\model is an agent based, spatially explicit model of mountain grassland. Individual plants compete on a torus grid for light and water. Competition is indirect and arises from resource split between individuals in the same cell, based on their exchange rates derived from their individual traits. Above- and below-ground resource assimilation results in production of organic matter that is then distributed between the different plant carbon pools according to a plastic allocation scheme. Fitness of individual plants, and so community dynamics, emerge from the species specific strategy and the impact of allocation of the newly produced OM, on growth, survival and reproduction.\\
\indent 
flexibility to address multiple questions on plasticity and coexistence, and extentionnality, strategy space, etc... but it comes with the cost of complexity, slowness and difficulty of parametrization.

%agent based model of mountain grassland\\
%spatially explicit competition for light and water\\
%multi-dimensional strategic space\\
%new plastic allocation mechanism between major carbon pools based on future estimation\\

\paragraph{General framework and carbon pools}
\indent Plants in \model are defined by their species, position, carbon pool size and expectations for the future conditions. Carbon pools are divided between preservation pools (i.e. storage and reproduction pools) and development pools that constitute stem, leaves and roots. Leaves and roots are the exchange surfaces of the plant and critical trade-of between their structure and function have been enlighten. LES, eventual root similar spectrum (paper in favour, paper againts). These trade-ofs can be partly explained by the structure of the organs that result from allocation between two distinct complementary carbon pools: active tissues and structural tissues (Shipley, Reich). Changes in the ratio of active over structural tissues induce changes in exchange area per mass unit (because of differences in density), overall respiration and lifespan, these relationships are detailed in figure \ref{fig:trade-of}.\\
\indent The species default values for these ratios (in leaves and roots) define the overall strategy of resource acquisition and use of the species (conservative or exploitative). In combination with other traits driving trade-of on reproductive (seed mass vs seed number), space occupancy (self shading vs vulnerability), phenology (early vs late flowering) strategies, they build a multidimensional space where each position is a strategy, shared by all individuals of the same species. This allows the model to run with generic parameters shared by all species, that translate the ubiquitous chemistry, physic and physiology rules in the system. Differences in growth rates, phenotypes and fitness only emerge from differences in the species strategies within the strategy space built by the functional trade-ofs.\\

\paragraph{Daily cycle and competition for resources}
\indent Cycle - pseudo code.\\
\indent Competition for light and water are handled in similar way. Light competition is computed pixel by pixel. Each pixel is divided in layer of homogeneous density between heights of two consecutive overlapping plants in the pixel. In each layer the potential photosynthesis is integrated based on average density and top incident light supposing light transmission follows Beer Lambert's law and an homogeneous vertical leaf area distribution. The potential photosynthesis is summed up for all layers of all pixel for each plants, and converted into total plant water demand thanks to WUE parameter.\\
\indent Water competition is also computed by pixel, where all water in the pixel is equally accessible to the root in the said pixel. The total plant water demand is divided between pixels of the rooting zone relatively to the mass proportion in the pixel. The pixel water demand may limit the water competition if the potential root water uptake in the focus pixel is greater than the water demand relative to the pixel. If total potential water uptake of plants in the pixel exceeds the available water, water absorption is then proportionally distributed with the potential water uptake of each plant compared to the total potential uptake.\\
\indent The growth production is computed from the realised water uptake and the water use efficiency. Maintenance respiration of only active tissues is subtracted to the growth production, and if positive the result is reduced by the growth respiration to compute the net primary production (NPP) converted in dry organic matter thanks to a constant tissue carbon content.\\

\paragraph{Allocation of carbon and plasticity}
Once produced, the organic matter is allocated to the stem to reach the minimal mechanical support. The remaining OM is divided between preservation pools, if the plant is mature, and plant leaves and root development. Only the leaves and roots pools are plastic, that means that the fractions allocated to the two pools of leaves and the two pools of roots can change over time. These changes lead to different shoot:root ratio and eventually different functional trait values. Active modifications of the size of the pools (increase only, no reallocation) are driven by the following hypothesis: \emph{plants adjust their phenotype to improve the carbon return in an hypothetical future estimated from genetic memory and individual experience, by particularly maintaining the equilibrium between above- and below-ground activity}. This hypothesis contains multiple key elements that we will try to explain in the following paragraphs.\\
\indent \\

\subparagraph{Plasticity concepts}
Perception - Projection - Adaptation

\subparagraph{The allocation algorithm}

\indent Perception in the model is directly derived from organ activity levels, and normalized over organs exchange area. In order to be used in a projection where the plant changes its phenotype, the proxy to resource availability have to be as independent from plant phenotype as possible. However there is clear dependency between leaf activity and root activity, indeed the water uptake is limited by the condition that the plant can transpire it. This condition is necessary to keep the water competition algorithm manageable and not model water storage as it can be done for succulent plants (Björn 2006). To approximate the potential water availability, the water uptake of the focus plant is recomputed considering no limitation by transpiration.\\
\indent Estimation of conditions is the key part in the plasticity algorithm because it fulfields two roles: the projection of conditions in a possible future, and part of the adaptation strategy. This double role results from limitation of the adaptation processes. As earlier argued, the adaptation of the plant to a future cannot consists in the maximization of a complex longterm-multiparameter fitness function. The plastic mechanism is reduced to the optimisation of some subfunctions, and this process should be shared by all plants, therefore, the control of the adapatation strategy stands in the parameters of these subfunctions, being the projection of the future. This duality is expressed in the formula of condition estimation. The projection part stands in the calculation of delta condition and the addition of this to the current estimation. The strategic aspect...


conditions of competition\\
future estimation: memory, maximize return over time...\\
objective functions\\
plasticity cost\\
algorithms

\section{Results}
\paragraph{One plant growth and development}
hypothesis: coordination, pot, same thickness.\\
fast-slow effect\\
different plasticity mechanisms effects on individual growth\\
plasticity - strategy interaction: plasticity capacity. 

\paragraph{Monoculture: competition and plasticity}
comparison of no plasticity - plasticity productivity\\
play with scales: time and space resolution.

\paragraph{Mixed community: competition}

\section{Discussion}


%_________________________________________________________________
%\section{From individual response to community response}
%\subsection{Individual niche, plasticity and strategy}
%What define the niche of a species, does the active/structural tissue trade-off affect niche ?
%How plasticity capacity affect niche ?
%--> 
%
%\subsection{Competitive ability and plasticity}
%What makes a species a (good invasive) a good competitor ? act/str ? -> role of plasticity: reactivity ? --> how to measure the competitive ability ? source of variation ? --> more or less coexistence
%
%invasion: compete, and stay -> reproduction, stress/disturbance resistance ?
%
%\subsection{Coexistence and diversity}
%explore niche overlapping and partitioning
%
%
%
%\section{Coexistence versus invasibility}
%\subsection{Sources of variation and effect on coexistence}
%\subsection{Mechanisms of coexistence}
%\subsection{Invasiveness }
%
%\section{Role of management}
%response to management ? --> affected species (fct of strategy, cons, time), recovery: plasticity


 \nocite{TitlesOn}
 \bibliographystyle{cell}
 \bibliography{biblio_draft}


  \end{document}
 }