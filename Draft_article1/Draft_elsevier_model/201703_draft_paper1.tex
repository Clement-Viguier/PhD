\documentclass[review]{elsarticle}

\usepackage{lineno,hyperref}
\modulolinenumbers[5]

\journal{Journal of \LaTeX\ Templates}

%%%%%%%%%%%%%%%%%%%%%%%
%% Elsevier bibliography styles
%%%%%%%%%%%%%%%%%%%%%%%
%% To change the style, put a % in front of the second line of the current style and
%% remove the % from the second line of the style you would like to use.
%%%%%%%%%%%%%%%%%%%%%%%

%% Numbered
%\bibliographystyle{model1-num-names}

%% Numbered without titles
%\bibliographystyle{model1a-num-names}

%% Harvard
%\bibliographystyle{model2-names.bst}\biboptions{authoryear}

%% Vancouver numbered
%\usepackage{numcompress}\bibliographystyle{model3-num-names}

%% Vancouver name/year
%\usepackage{numcompress}\bibliographystyle{model4-names}\biboptions{authoryear}

%% APA style
%\bibliographystyle{model5-names}\biboptions{authoryear}

%% AMA style
%\usepackage{numcompress}\bibliographystyle{model6-num-names}

%% `Elsevier LaTeX' style
\bibliographystyle{elsarticle-num}
%%%%%%%%%%%%%%%%%%%%%%%

%% Custom commands:
\newcommand{\model}{\textbf{\textit{MountGrass}}~}




%%%%%%%%%%%%%%%%%%%%%%%

\begin{document}

\begin{frontmatter}

\title{Elsevier \LaTeX\ template\tnoteref{mytitlenote}}
\tnotetext[mytitlenote]{Fully documented templates are available in the elsarticle package on \href{http://www.ctan.org/tex-archive/macros/latex/contrib/elsarticle}{CTAN}.}

%% Group authors per affiliation:
\author{Elsevier\fnref{myfootnote}}
\address{Radarweg 29, Amsterdam}
\fntext[myfootnote]{Since 1880.}

%% or include affiliations in footnotes:
\author[mymainaddress,mysecondaryaddress]{Elsevier Inc}
\ead[url]{www.elsevier.com}

\author[mysecondaryaddress]{Global Customer Service\corref{mycorrespondingauthor}}
\cortext[mycorrespondingauthor]{Corresponding author}
\ead{support@elsevier.com}

\address[mymainaddress]{1600 John F Kennedy Boulevard, Philadelphia}
\address[mysecondaryaddress]{360 Park Avenue South, New York}

\begin{abstract}
\end{abstract}

\begin{keyword}
\texttt{elsarticle.cls}\sep \LaTeX\sep Elsevier \sep template
\MSC[2010] 00-01\sep  99-00
\end{keyword}

\end{frontmatter}

\linenumbers

\section{Introduction}
\indent One of the main challenges of community ecology has been to explain the observed diversity despite few theoretical supports to high coexistence, as know as the "plankton paradox". During the last 25 years, our theoretical understanding of this apparent paradox is nourished by empirical and modelling work. This progress partly arose from a shift between approaches focusing on species to the use of functional traits. Functional traits allows the investigation of broader patterns and mechanisms shaping natural communities. Traits emerged s a great tool to link theoretical concepts to empirical studies. The rise of large trait datasets and online trait data repositories favour the investigation of broad general patterns of response to climate, trade-offs, community composition. This shift also introduced the concept on functional diversity that focused more on the functional role of the species in the ecosystem than on their identity.\\
% talk also about use of traits to niche filtering and climatic niche. Response to climate. Interactions.
%MRH etc, see nequist for inspiration. + lavorel and diaz. In addition to a better understanding of ecological interactions, plant functional traits are the basis of the assessment of ecosystem services.\\
\indent Despite the advances made in community ecology by the use of functional trait, certain mechanisms are still misunderstood and the mysteries of coexistence are not yet resolved. This limitation of the approach may lie in the use of the species average traits of species. According to Violle and al. (check the ref) intra-specific trait variability (ITV)... Intra-specific trait variability arises from three main mechanisms (see box 1 for details). Incorporating this intra-specific variability in community models in necessary to enhance our understanding of community assemblage rules and community responses to biotic and abiotic drivers. ITV has been shown to have contrasted effects on conclusions we can make from community models. Jung et al. demonstrated the positive effect of intra-specific variability on coexistence by mitigating the abiotic filtering effect and improving the niche partitioning effect (check that). In the other hand the ITV are supposed to reduce predicted coexistence through Jensen inequality effect (on both XXX function and competitive effects, see for details), despite a light positive impact of released intra-specific competition of low density species. The relative impact of these stabilizing (better niche partitioning, lower intra-specific competition at low density, etc...) and destabilizing () effects of intra-specific variations cannot be resolved by the use of theoretical models only. Empirical works show interesting correlations between shoot plasticity and diversity, but this relationship is only explained by an increase in ramet density and not a increase in diversity per ramet (sign of increased niche partitioning). Disentangle effects, complexity, tractable, generalization... Differences of response 
Between the complexity of natural systems and the oversimplification of theoretical models, there is a gap to fill if we want to better understand the role of phenotypic plasticity as a regulator of plant responses to climate drivers and biological interactions.\\

\indent To attempt to fill the gap between empirical approaches and theoretical we try to address the following questions: 
%can the diversity of plastic strategies be reproduced by a generic model of phenotypic plasticity ?
Can a simple shared plasticity functioning explain the diversity of plant plastic allocation strategies ?
How the phenotypic plasticity alter species potential niches and species interactions ?
What is the impact of plasticity costs on the plant strategies and interactions ?\\

\indent In order to tackle these questions we developed a generic process-based model of mountain grasslands. This model 
rely on a generic framework of plant functioning based on multidimensional strategies and allocation trade-offs, extended with phenotypic plasticity. The model is mostly in this paper at the individual scale to settle a good understanding of phenotypic plasticity at the lowest scale. Work at the higher scale will be able to extend from this basis thanks to community scale simulations. Hopefully this ensemble of work will nourish the discussion about the role of intra-specific variation and phenotypic plasticity on community dynamics, response to global changes and evolution.

%\section{Theory}
%\subsection{Multidimensional strategy space}
%
%\subsection{Allocation trade-offs}
%\subsection{Phenotypic plasticity: between parental memory and confidence in individual experience}

\section{Methods}
\subsection{Purpose}
The model \model is designed to investigate the effect of phenotypic plasticity on coexistence mechanisms in response to climatic and management drivers on mountain grasslands individual plants and communities. Its generic structure allows the representation of numerous species with diverse strategies. Its design also let us use it both as a toy model for plants in controlled conditions (main use case in this paper) or as a community model.\\

\subsection{Structure}
The model is an individual, spatially explicit, process-based model.\\
Plants in \model are represented as two cylinders: one for shoot and one of fix depth (soil depth) for roots. These basic shapes are used in the calculation of resource gathering. Individuals are defined by an ensemble of species specific traits (see table ) and state variables (table). Plant phenotype can be derived from model parameters, species specific traits and size of the different carbon pools. The plant phenotype is completed by an ensemble of variables that describe its internal state and strategy. The carbon pool decomposition relies on the general patterns of the Leaf Economic Spectrum and other large scale studies. The limits of the use of general patterns at the local scale in discussed further in this paper. Statistical modelling approach by Shipley supports the role of one particular trade-off to explain the LES: the proportion of cell tissue (related to cell size and number) to cell-wall tissue (cell-wall thickness). This assumption that the allocation between structural (cell-wall) and active (cell) tissues can explain main trade-offs is at the center on the model carbon pool decomposition described in figure . Other carbon pools are: stem, storage and reproduction. %\\
Plants : multiple strategies,2 cylinders, 4 carbon pools.\\
Environment : soil (water parameters, depth), atmosphere (temp, humidity, light)\\
scales and processes.

\subsection{Representation of processes}
Resources gathering and competition\\
Resource allocation\\
Plasticity (and costs).\\
(reproduction ?)\\
\subsection{Initialisation and inputs}
Because of implementation limitations running sensitivity analysis and full calibration is not possible. The model was used with fixed parameter values unless explicitly said (e.g. for plasticity costs analysis) (see appendix for details). The model was mostly used in the following set-ups:
\begin{itemize}
\item[Pot simulation:]
\item[Double pot simulation:]
\item[Greenhouse simulation:]
%\item[Grassland simulation]
\end{itemize}
In the rest of this paper we will use the terms "pot simulation", "double pot simulation" and "greenhouse simulation" to refer  to these set-ups.\\
If not specified or varying the default precipitation and radiance levels where as follow: 2mm per day and 300 Watt per day (10/12 hours). Species traits are default traits (see appendix for details)\\
Maturity and investment to storage or reproductive tissues is ignored to use the total growth as proxy for fitness.

\section{Simulation experiments}
\subsection{Benefits of plasticity on single plant growth}
The effect of plasticity on single plant growth is explore with pot simulations in different conditions. The relative effect of climate and projection accuracy are analysed.

%The advantage of phenotypic plasticity is analysed in a varying environment.

\subsection{Strategic differentiation and convergence}
The model links species memory to species default phenotype (see method and appendices for details) but plasticity releases the control of the individual phenotype by the species traits. Because all plants share the same objective (at least in the model by optimisation of the gain function), trait convergence is expected. The convergence of traits is explore as a function of: the confidence on condition memory, cost of phenotype (check ref and word-use) and variability. 


\subsection{Potential and realised niche}

\section{Results}
\subsection{Individual plant growth and phenotypic plasticity benefits}
Lag effect. Variable effect could be studied but already studied when plasticity is advantageous. Our model should go in that direction.\\
Plasticity cost\\

\subsection{Strategic differentiation and convergence}

\subsection{Potential and realised niche}


\section*{References}
\cite{Dirac1953888}
\bibliography{201703_bib_paper1}

\part*{Appendices}

%\section{Theory}

\section{\model description}

\section{State variables, traits and parameters}
\subsection{State variables}
\subsection{Species specific traits}
\subsection{Parameters}

\section{Simulations}


\end{document}